\documentclass[aspectratio=1610]{beamer}
\usepackage[utf8]{inputenc}
\usepackage{ragged2e}
\usepackage{xcolor}
\usepackage[italian]{babel}
\usepackage{multirow}
\usetheme[progressbar=frametitle,titleformat=smallcaps]{metropolis}
\setbeamertemplate{frame numbering}[fraction]
\setbeamercovered{dynamic}
\definecolor{rosso}{RGB}{255, 0, 0}
\definecolor{giallo}{RGB}{254,212,23}
\hypersetup{colorlinks=true,linkcolor=black,urlcolor=rosso}
\setbeamercolor{palette primary}{fg=black, bg=giallo}
\setbeamercolor{background canvas}{bg=white}
\setbeamercolor{normal text}{fg=black}
\setbeamercolor{progress bar}{fg=rosso}
\setbeamercolor{framesubtitle}{fg=rosso}
\setbeamercolor{normal text .dimmed}{fg=giallo}
\setbeamercolor{block title alerted}{fg=rosso, bg=giallo}
\setbeamerfont{caption}{size=\tiny}
\setbeamerfont{caption name}{size=\tiny}
\setlength{\abovecaptionskip}{0pt}
\makeatletter
\metroset{block=fill}
\setlength{\metropolis@progressinheadfoot@linewidth}{1pt} 
\setlength{\metropolis@progressonsectionpage@linewidth}{1pt}
\setlength{\metropolis@titleseparator@linewidth}{1pt}
\makeatother

\title{TITOLO}
\subtitle{sottotitolo}
\date{}
\institute{\textit{
        Fonti:
        \begin{itemize}
            \item[-] \href{www.nomefonte1.it}{NOME FONTE 1}
            \item[-] \href{www.nomefonte2.it}{NOME FONTE 2}
        \end{itemize}
    }
}

\begin{document}

\begin{frame}[plain, noframenumbering]
    \titlepage
\end{frame}

\section{SEZIONE 1: IMMAGINE CENTRATA}

\begin{frame}{TITOLO SLIDE}
%    \begin{columns}
%        \column{.5\textwidth}
%        \begin{figure}
%            \includegraphics[width=\linewidth]{img/immagine.png}
%            \caption{{creata con \href{LINK SOFTWARE DI CREAZIONE}{NOME SOFTWARE}}}
%        \end{figure}
%    \end{columns}
\end{frame}

\section{SEZIONE 2: ELENCO PUNTATO}

\begin{frame}{TITOLO SLIDE}
%    \begin{itemize}
%        \item \textbf{NOME}: descrizione;
%        \pause
%        \item \textbf{NOME}: descrizione;
%        \pause
%        \item \textbf{NOME}: descrizione;
%    \end{itemize}
\end{frame}

\section{SEZIONE 3: TABELLA}

\begin{frame}{TITOLO SLIDE}
%    \centering
%    \begin{tabular}{c||c}
%        \textbf{TITOLO COLONNA 1} & \textbf{TITOLO COLONNA 2} \\
%        \hline
%        \hline
%        \pause
%        ELEMENTO SINISTRA & ELEMENTO DESTRA \\
%        \hline
%        \pause
%        ELEMENTO SINISTRA & ELEMENTO DESTRA \\
%        \hline
%    \end{tabular}
\end{frame}

\section{SEZIONE 4: SEQUENZA IMMAGINI}

\begin{frame}{TITOLO SLIDE}
%   \only<1 | handout:0>{\begin{figure}
%       \includegraphics[width=\linewidth]{img/immagine1.png}
%       \caption{{creata con \href{LINK SOFTWARE DI CREAZIONE}{NOME SOFTWARE}}}
%   \end{figure}}
%   \only<2 | handout:1>{\begin{figure}
%       \includegraphics[width=\linewidth]{img/immagine2.png}
%       \caption{{creata con \href{LINK SOFTWARE DI CREAZIONE}{NOME SOFTWARE}}}
%   \end{figure}}
%   \only<3 | handout:2>{\begin{figure}
%       \includegraphics[width=\linewidth]{img/immagine3.png}
%       \caption{{creata con \href{LINK SOFTWARE DI CREAZIONE}{NOME SOFTWARE}}}
%   \end{figure}}
\end{frame}

\section{SEZIONE 5: DEFINIZIONE}

\begin{frame}{TITOLO SLIDE}
    \begin{alertblock}{DEFINIZIONE}
        \begin{minipage}{0.98\linewidth}
            \justifying
            % testo definizione\\
            %\bigskip
            %\tiny{\textbf{Curiosità}}\\
            %\tiny{\href{Link curiosità}{Titolo curiosità}}
        \end{minipage}
    \end{alertblock}
\end{frame}

\section{SEZIONE 6: DUE COLONNE (testo, immagine)}

\begin{frame}{TITOLO SLIDE}
    \begin{columns}
        \column{.5\textwidth}
            \justifying
%           testo a sinistra dell'immagine
%           \bigskip
%           \tiny{\textbf{curiosità}}\\
%           \tiny{LINK CURIOSITA'}{TITOLO CURIOSITA'}}
        \column{.5\textwidth}
%           \begin{figure}
%               \includegraphics[width=\linewidth]{img/immagine1.png}
%               \caption{{creata con \href{LINK SOFTWARE DI CREAZIONE}{NOME SOFTWARE}}}
%           \end{figure}
    \end{columns}
\end{frame}

\end{document}