\documentclass[aspectratio=1610]{beamer}
%\documentclass[aspectratio=1610, handout]{beamer}
\usepackage[utf8]{inputenc}
\usepackage{ragged2e}
\usepackage{xcolor}
\usepackage[italian]{babel}
\usepackage{multirow}
\usetheme[progressbar=frametitle,titleformat=smallcaps]{metropolis}
\setbeamertemplate{frame numbering}[fraction]
\setbeamercovered{dynamic}
\definecolor{rosso}{RGB}{255, 0, 0}
\definecolor{giallo}{RGB}{254,212,23}
\hypersetup{colorlinks=true,linkcolor=black,urlcolor=rosso}
\setbeamercolor{palette primary}{fg=black, bg=giallo}
\setbeamercolor{background canvas}{bg=white}
\setbeamercolor{normal text}{fg=black}
\setbeamercolor{progress bar}{fg=rosso}
\setbeamercolor{framesubtitle}{fg=rosso}
\setbeamercolor{normal text .dimmed}{fg=giallo}
\setbeamercolor{block title alerted}{fg=rosso, bg=giallo}
\setbeamerfont{caption}{size=\tiny}
\setbeamerfont{caption name}{size=\tiny}
\setlength{\abovecaptionskip}{0pt}
\makeatletter
\metroset{block=fill}
\setlength{\metropolis@progressinheadfoot@linewidth}{1pt} 
\setlength{\metropolis@progressonsectionpage@linewidth}{1pt}
\setlength{\metropolis@titleseparator@linewidth}{1pt}
\makeatother

\title{FAKE NEWS, DEEP FAKE E REVENGE PORN}
\subtitle{informazione libera, identificare e proteggersi dai fake e rimozione di contenuti indesiderati}
\date{}
\institute{\textit{
        Fonti:
        \begin{itemize}
            \item[-] \href{https://attivissimo.blogspot.com/2023/05/podcast-rsi-story-panico-per-la-guerra.html}{Il Disinformatico}
            \item[-] \href{https://www.garanteprivacy.it/home/docweb/-/docweb-display/docweb/9845178}{Cerrina Feroni (Garante privacy): ``Cosa fare contro la dittatura del pensiero dominante''} 
            \item[-] \href{https://www.garanteprivacy.it/temi/revengeporn}{Garante per la protezione dei dati personali}
            \item[-] \href{https://lucysullacultura.com/linformazione-sta-sparendo-dai-social-ed-e-un-problema/}{Andrea Daniele Signorelli (Lucysullacultura)} 
            \item[-] \href{https://www.ictsecuritymagazine.com/articoli/deepfake-e-intelligenza-artificiale-tra-rischi-di-sicurezza-e-vantaggi/}{Riccardo Laurenti (ICT Security Magazine)} 
        \end{itemize}
    }
}

\begin{document}

\begin{frame}[plain, noframenumbering]
    \titlepage
\end{frame}

\section{FAKE NEWS}

\begin{frame}{FAKE NEWS STORICA}
    \begin{alertblock}{LA GUERRA DEI MONDI}
        \begin{minipage}{0.98\linewidth}
            \justifying
            Sono le otto di sera della vigilia di Halloween del 1938. 
            Negli Stati Uniti, la rete radiofonica Columbia Broadcasting Systems (CBS), 
            trasmette un adattamento del libro di Herbert George Wells ``La Guerra dei Mondi''. 
            Il radiodramma inizia come se fosse un programma musicale e poi sembra interrompersi per 
            un notiziario urgente:
        \end{minipage}
    \end{alertblock}
    \pause
    \begin{alertblock}{}
        \begin{minipage}{0.98\linewidth}
            \justifying
            \textit{Signore e signori, vogliate scusarci per l'interruzione del nostro programma 
            di musica da ballo, ma ci è appena pervenuto uno speciale bollettino della 
            Intercontinental Radio News. Alle 7:40, ora centrale, il professor Farrell 
            dell'Osservatorio di Mount Jennings, Chicago, Illinois, ha rilevato diverse 
            esplosioni di gas incandescente che si sono succedute ad intervalli regolari 
            sul pianeta Marte. Le indagini spettroscopiche hanno stabilito che il gas 
            in questione è idrogeno e si sta muovendo verso la Terra ad enorme velocità.}\\
            \bigskip
            \tiny{\textbf{``La Guerra dei Mondi'' di Orson Welles}}\\
            \tiny{\href{https://www.youtube.com/watch?v=Xs0K4ApWl4g}{Orson Welles - War Of The Worlds - Radio Broadcast 1938 - Complete Broadcast.}}\\
        \end{minipage}
    \end{alertblock}
\end{frame}

\begin{frame}{FAKE NEWS}
    \begin{alertblock}{MISINFORMATION}
        \begin{minipage}{0.98\linewidth}
            \justifying
            Si verifica quando contenuti falsi vengono diffusi accidentalmente da 
            utenti inconsapevoli, convinti di divulgare contenuti che corrispondono 
            a un reale interesse sociale.\\
            \bigskip
            \tiny{\textbf{Come proteggersi?}}\\
            \tiny{\href{https://www.newsguardtech.com/it/}{Newsguardtech.com}}\\
        \end{minipage}
    \end{alertblock}
    \pause
    \begin{alertblock}{MALINFORMATION}
        \begin{minipage}{0.98\linewidth}
            \justifying
            Si verifica quando informazioni veritiere vengono pubblicate 
            con lo specifico scopo di creare conseguenze avverse.
        \end{minipage}
    \end{alertblock}
    \pause
    \begin{alertblock}{DISINFORMATION}
        \begin{minipage}{0.98\linewidth}
            \justifying
            Si verifica quando il contenuto della notizia diffusa è 
            intenzionalmente falso e trasmesso ad hoc per causare conseguenze dannose.\\
            \bigskip
            \tiny{\textbf{Come proteggersi?}}\\
            \tiny{\href{https://www.bufale.net/}{Bufale.net}}\\
        \end{minipage}
    \end{alertblock}
\end{frame}

\begin{frame}{SOCIAL NETWORK COME FONTE DI INFORMAZIONE}
    \begin{itemize}
        \justifying
        \item \textbf{RICERCA ALGORITMICA PASSIVA}: la ricerca dell’informazione sui social non è quasi mai attiva: 
        le notizie compaiono sul nostro feed a seconda di quanto l’algoritmo ci ritenga interessati a politica, 
        temi sociali, ecc (o quanto l’abbiamo addestrato a farlo, seguendo varie testate, giornalisti, 
        divulgatori e così via);
        \pause
        \item \textbf{SHADOW BAN (``divieto ombra'')}: meccanismo con cui le piattaforme social 
        riducono drasticamente - e senza dichiararlo - la visibilità di alcuni utenti e dei contenuti 
        che producono;
        \pause
        \item \textbf{OPT OUT CONTENUTI POLITICI}: Meta nella sezione “Privacy” ha inserito un pulsante 
        - attivato di default - che ``limita i contenuti di natura politica''. Instagram si limita poi a 
        definire, con vaghezza, ``politici'' i contenuti che ``potrebbero menzionare governi, elezioni o 
        argomenti sociali che interessano un gruppo di persone e/o la società in generale''.
    \end{itemize}
    \bigskip
    \tiny{\textbf{E i gionali/siti/app di news online?}}\\
    \tiny{\href{https://www.feltrinellieducation.it/magazine/l-economia-dell-attenzione-e-il-paradosso-che-sta-uccidendo-i-giornali}{L’economia dell’attenzione}}\\
\end{frame}

\begin{frame}{SOCIAL NETWORK COME FONTE DI INFORMAZIONE}
    \begin{columns}
        \column{.6\textwidth}
            \justifying
            \textbf{Social networking che non è in vendita}.\\
            La tua home feed dovrebbe essere riempita con ciò che conta di più per te, 
            non con ciò che una corporazione pensa che dovresti vedere. Social media 
            radicalmente diverso, di nuovo nelle mani delle persone.\\
            \bigskip
            \tiny{\textbf{MASTODON}}\\
            \tiny{\href{https://joinmastodon.org/it}{Cos'è e come funziona Mastodon?}}\\
        \column{.4\textwidth}
            \begin{figure}
                \includegraphics[width=\linewidth]{img/mastodon.png}
                \caption{{Fonte: \href{https://joinmastodon.org/it}{Mastodon}}}
            \end{figure}
    \end{columns}
\end{frame}

\begin{frame}{FEED RSS}
    \begin{columns}
        \column{.6\textwidth}
            \justifying
            l'\textbf{RSS} è uno dei più popolari formati per la distribuzione di contenuti Web. 
            L'applicazione principale per cui è noto sono i flussi che permettono di \textbf{essere aggiornati su 
            nuovi articoli o commenti pubblicati nei siti di interesse senza doverli visitare manualmente 
            a uno a uno}. Visto che il formato è predefinito, un qualunque lettore RSS potrà presentare 
            in una maniera omogenea notizie provenienti dalle fonti più diverse.\\
            \bigskip
            \tiny{\textbf{RSS Feed Reader}}\\
            \tiny{\href{https://f-droid.org/it/packages/com.nononsenseapps.feeder/}{Feeder}}
        \column{.4\textwidth}
            \begin{figure}
                \includegraphics[width=\linewidth]{img/rss.png}
                \caption{{Fonte: \href{https://commons.wikimedia.org/wiki/File:Rss-feed.svg}{Wikimedia Commons}}}
            \end{figure}
    \end{columns}
\end{frame}

\section{IA E FAKE NEWS}

\begin{frame}{IA E FAKE NEWS}
    \begin{columns}        
        \column{.4\textwidth}
            \begin{figure}
                \href{https://www.youtube.com/watch?v=oxXpB9pSETo}{\includegraphics[width=\linewidth]{img/play.png}}
                \caption{{Fonte: \href{https://www.youtube.com/watch?v=oxXpB9pSETo}{Diep Nep: This is not Morgan Freeman}}}
            \end{figure}
        \column{.6\textwidth}
            \begin{alertblock}{DEEP FAKE}
                \begin{minipage}{0.98\linewidth}
                    \justifying
                    I \textbf{deepfake sono contenuti multimediali manipolati}, spesso video, \textbf{che utilizzano la tecnologia 
                    dell’intelligenza artificiale} per creare immagini realistiche di persone o situazioni che 
                    in realtà non sono accadute.\\
                    \bigskip
                    \tiny{\textbf{Whitepaper (06/2022) di Francesco Arruzzoli}}\\
                    \tiny{\href{https://www.ictsecuritymagazine.com/pubblicazioni/deepfake-cyber-intelligence/}{Deepfake \& cyber intelligence}}\\
                    \tiny{\textbf{Deepfakelab}}\\
                    \tiny{\href{https://deepfakelab.theglassroom.org/index-it_IT.html}{Deepfakelab: come riconoscere i deepfake}}
                \end{minipage}
            \end{alertblock}
            \pause
            \begin{alertblock}{QUIZ}
                \begin{minipage}{0.98\linewidth}
                    \tiny{\href{https://detectfakes.kellogg.northwestern.edu/}{AI-GENERATED OR REAL?}}
                \end{minipage}
            \end{alertblock}
    \end{columns}
\end{frame}

\begin{frame}{ALCUNE TIPOLOGIE DI DEEP FAKE}
    \begin{itemize}
        \justifying
        \item \textbf{FACE SWAPS}: il viso di una persona di origine (source) viene 
        sovrapposto su una persona destinataria (target).
        \pause
        \item \textbf{LIP-SYNCING}: un video viene modificato per far sembrare che la persona 
        destinataria pronunci frasi fittizie. Questo processo coinvolge l’utilizzo della voce 
        registrata da una o più fonti, sincronizzata con un altro video per renderlo autentico. 
        \pause
        \item \textbf{PUPPET MASTER}: imposizione dell’aspetto di una persona destinataria (puppet) 
        su un’altra persona che compie azioni.\\
        \tiny{\textbf{Curiosità}}\\
        \tiny{\href{https://www.netflix.com/it/title/81123425}{Verso il futuro, espisodio 7: La vita dopo la morte}}\\
        \pause
        \normalsize
        \item \textbf{DEEP FAKE AUDIO}: manipola e genera file audio falsi che sembrano autentici. 
        Questa tecnologia può essere utilizzata per alterare voci, creare discorsi o dialoghi inventati, 
        o persino imitare voci di persone specifiche in modo convincente.\\
        \tiny{\textbf{Curiosità}}\\
        \tiny{\href{https://www.wired.it/article/alexa-imita-voci-morti}{Imitazioni di Alexa}}\\
        \pause
        \normalsize
        \item \textbf{DEEP NUDE}: le immagini vengono manipolate con lo scopo di rimuovere 
        gli indumenti di una persona, creando immagini finte e sessualmente esplicite.
    \end{itemize}
\end{frame}

\section{REVENGE PORN}

\begin{frame}{REVENGE PORN}
    \begin{alertblock}{DEFINIZIONE}
        \begin{minipage}{0.98\linewidth}
            \justifying
            Il revenge porn, consiste nell’invio, consegna, cessione, pubblicazione o diffusione, 
            da parte di chi li ha realizzati o sottratti e \textbf{senza il consenso della persona} cui si riferiscono, 
            di immagini o video a contenuto sessualmente esplicito destinati a rimanere privati. 
            Tale diffusione avviene di solito a scopo vendicativo, per denigrare pubblicamente, 
            ricattare, bullizzare o molestare. Si tratta quindi di una \textbf{pratica che può avere effetti 
            drammatici a livello psicologico, sociale e anche materiale sulla vita delle persone che ne sono vittime}.\\
            \bigskip
            \tiny{\textbf{Come segnalare un abuso?}}\\
            \tiny{\href{https://servizi.gpdp.it/diritti/s/revenge-porn-scelta-auth}{Segalazione al Garante per la protezione dei dati personali}}\\
        \end{minipage}
    \end{alertblock}
\end{frame}

\begin{frame}{COME DIFENDERSI?}
    \begin{minipage}{0.98\linewidth}
        \centering
        \huge
        \textit{``LA PRIMA E PI\'U IMPORTANTE FORMA DI DIFESA SONO SEMPRE LA CONSAPEVOLEZZA E LA PRUDENZA.''}\\
    \end{minipage}\\
    \bigskip
    \tiny{\textbf{Come rimuovere i contenuti?}}\\
    \tiny{\href{https://takeitdown.ncmec.org/it/}{Take It Down (servizio per minorenni)}}\\
    \tiny{\href{https://stopncii.org/}{Stop Non-Consensual Intimate Image Abuse (servizio per maggiorenni)}}
\end{frame}
\end{document}