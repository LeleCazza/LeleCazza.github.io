\documentclass[aspectratio=1610]{beamer}
%\documentclass[aspectratio=1610, handout]{beamer}
\usepackage[utf8]{inputenc}
\usepackage{ragged2e}
\usepackage{xcolor}
\usepackage[italian]{babel}
\usepackage{multirow}
\usetheme[progressbar=frametitle,titleformat=smallcaps]{metropolis}
\setbeamertemplate{frame numbering}[fraction]
\setbeamercovered{dynamic}
\definecolor{rosso}{RGB}{255, 0, 0}
\definecolor{giallo}{RGB}{254,212,23}
\hypersetup{colorlinks=true,linkcolor=black,urlcolor=rosso}
\setbeamercolor{palette primary}{fg=black, bg=giallo}
\setbeamercolor{background canvas}{bg=white}
\setbeamercolor{normal text}{fg=black}
\setbeamercolor{progress bar}{fg=rosso}
\setbeamercolor{framesubtitle}{fg=rosso}
\setbeamercolor{normal text .dimmed}{fg=giallo}
\setbeamercolor{block title alerted}{fg=rosso, bg=giallo}
\setbeamerfont{caption}{size=\tiny}
\setbeamerfont{caption name}{size=\tiny}
\setlength{\abovecaptionskip}{0pt}
\makeatletter
\metroset{block=fill}
\setlength{\metropolis@progressinheadfoot@linewidth}{1pt} 
\setlength{\metropolis@progressonsectionpage@linewidth}{1pt}
\setlength{\metropolis@titleseparator@linewidth}{1pt}
\makeatother

\title{SERVIZI PER IL CITTADINO}
\subtitle{Servizi digitali per il Cittadino e protezione dell’Identità Digitale}
\date{}
\institute{\textit{
        Fonti:
        \begin{itemize}
            \item[-] \href{https://digitalepopolare.it/breve-storia-della-digitalizzazione-della-pubblica-amministrazione}{Digitale Popolare}
            \item[-] \href{https://www.luisatreccani.it/legge-bassanini/}{Luisa Treccani, esperta in materia di legislazione e normativa scolastica}
        \end{itemize}
    }
}

\begin{document}

\begin{frame}[plain, noframenumbering]
    \titlepage
\end{frame}

\section{EVOLUZIONE DELLA PA}

\begin{frame}{Evoluzione della Pubblica Amministrazione in chiave digitale}
    \begin{itemize}
        \item \textbf{Anni '90}: Introduzione dei primi computer e software gestionali nelle amministrazioni 
        pubbliche, inizio del processo di informatizzazione;
        \pause
        \item \textbf{Decreto 59/1997 (riforma Bassanini)}: Riforma del sistema amministrativo, per sburocratizzare 
        l’amministrazione pubblica e i procedimenti amministrativi, renderli più efficienti, più snelli e in grado di offrire 
        servizi di qualità prevedendo l’adozione di tecnologie informatiche;
        \pause
        \item \textbf{Decreto 82/2005 Codice dell'Amministrazione Digitale}: Il \href{https://www.agid.gov.it/it/agenzia/strategia-quadro-normativo/codice-amministrazione-digitale}{CAD} 
        stabilì i principi e le regole per l’uso delle tecnologie digitali nella PA, ponendo le basi per 
        l’interoperabilità e la dematerializzazione dei documenti.
        \pause
        \item \textbf{2012: nascita di AgID (Agenzia per l'Italia Digitale)}: Agenzia che ha il compito di coordinare e 
        promuovere la digitalizzazione della PA, monitorando l’attuazione delle politiche digitali e assicurando 
        la coerenza degli interventi.
    \end{itemize}
\end{frame}

\section{AgID PIATTAFORME E TECNOLOGIE}

\begin{frame}{AGID}
    \begin{columns}
        \column{.5\textwidth}
        \begin{figure}
            \href{https://www.agid.gov.it/it}{\includegraphics[width=\linewidth]{img/agid.png}}
            \caption{{creata con \href{https://chatgpt.com/}{ChatGPT}}}
        \end{figure}
    \end{columns}
\end{frame}

\end{document}