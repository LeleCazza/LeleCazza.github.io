% FONTE TEMA https://github.com/matze/mtheme
\documentclass[aspectratio=1610]{beamer}
%\documentclass[aspectratio=1610, handout]{beamer}
\usepackage[utf8]{inputenc}
\usepackage{ragged2e}
\usepackage{xcolor}
\usepackage[italian]{babel}
\usepackage{multirow}
\usepackage{silence}
\WarningFilter{beamer}{}
\WarningFilter{metropolis}{}
\usetheme[progressbar=frametitle,titleformat=smallcaps]{metropolis}
\setbeamertemplate{frame numbering}[fraction]
\setbeamercovered{dynamic}
\definecolor{rosso}{RGB}{255, 0, 0}
\definecolor{giallo}{RGB}{254,212,23}
\hypersetup{colorlinks=true,linkcolor=black,urlcolor=rosso}
\setbeamercolor{palette primary}{fg=black, bg=giallo}
\setbeamercolor{background canvas}{bg=white}
\setbeamercolor{normal text}{fg=black}
\setbeamercolor{progress bar}{fg=rosso}
\setbeamercolor{framesubtitle}{fg=rosso}
\setbeamercolor{normal text .dimmed}{fg=giallo}
\setbeamercolor{block title alerted}{fg=rosso, bg=giallo}
\setbeamerfont{caption}{size=\tiny}
\setbeamerfont{caption name}{size=\tiny}
\setlength{\abovecaptionskip}{0pt}
\makeatletter
\metroset{block=fill}
\setlength{\metropolis@progressinheadfoot@linewidth}{1pt} 
\setlength{\metropolis@progressonsectionpage@linewidth}{1pt}
\setlength{\metropolis@titleseparator@linewidth}{1pt}
\makeatother

\title{IMPATTI AMBIENTALI DELLE TECNOLOGIE DIGITALI}
\subtitle{Crisi climatica e sostenibilità}
\date{}
\institute{\textit{
        Fonti:
        \begin{itemize}
            \item[-] \href{https://www.ibs.it/ne-intelligente-ne-artificiale-lato-libro-kate-crawford/e/9788815294197}{Né intelligente né artificiale. Il lato oscuro dell'IA}
            \item[-] \href{https://it.wikipedia.org/wiki/Antropocene}{Wikipedia}
        \end{itemize}
    }
}

\begin{document}

\begin{frame}[plain, noframenumbering]
    \titlepage
\end{frame}

\section{CLOUD E DATA CENTER}

\begin{frame}{CLOUD E DATA CENTER}
    \begin{columns}
        \column{.5\textwidth}
            \begin{alertblock}{DEFINIZIONE}
                \begin{minipage}{0.96\linewidth}
                    \justifying
                    Il \textbf{cloud} consente agli utenti di accedere agli stessi file e alle stesse 
                    applicazioni da ogni dispositivo \textbf{tramite la rete Internet}, perché l'elaborazione e 
                    l'archiviazione hanno luogo in \textbf{server} che si trovano in \textbf{data center}, 
                    invece che localmente, nel dispositivo dell'utente.\\
                    \bigskip
                    \tiny{\textbf{Approfondimento}}\\
                    \tiny{\href{https://www.cloudflare.com/it-it/learning/cloud/what-is-the-cloud/}{Cos'è il cloud secondo Cloudflare}}
                \end{minipage}
            \end{alertblock}
        \column{.5\textwidth}
            \begin{figure}
                \includegraphics[width=.8\linewidth]{img/cloud_1.png}
                \caption{{creata con \href{www.chatgpt.com}{ChatGPT}}}
            \end{figure}
    \end{columns}
\end{frame}

\begin{frame}{CLOUD E DATA CENTER}
    \begin{columns}
        \column{.5\textwidth}
            \begin{alertblock}{DEFINIZIONE}
                \begin{minipage}{0.96\linewidth}
                    \justifying
                    \textit{``Il \textbf{cloud} è una tecnologia estrattiva ad \textbf{alta intensità di risorse} che 
                    \textbf{converte l'acqua e l'elettricità in potenza computazionale}, lasciando dietro di 
                    sé una quantità considerevole di \textbf{danni ambientali che poi cela alla vista}''.}\\
                    \bigskip
                    \tiny{\textbf{Citazione}}\\
                    \tiny{\href{https://www.tunghui.hu/books/a-prehistory-of-the-cloud}{Tung-Hui Hu: A Prehistory of the Cloud}}
                \end{minipage}
            \end{alertblock}
        \column{.5\textwidth}
            \begin{figure}
                \includegraphics[width=.8\linewidth]{img/cloud_2.png}
                \caption{{creata con \href{www.chatgpt.com}{ChatGPT}}}
            \end{figure}
    \end{columns}
\end{frame}

\begin{frame}{CLOUD E DATA CENTER}
    \begin{figure}
        \includegraphics[width=\linewidth]{img/digitalizzazione.png}
        \caption{{fonte \href{https://luissuniversitypress.it/inquinamento-cloud-articolo-andrea-daniele-signorelli-digitalizzazione-lmdp/}{Andrea Daniele Signorelli: Inquinamento cloud}}}
    \end{figure}
\end{frame}

\begin{frame}{CLOUD E DATA CENTER}
    \begin{figure}
        \includegraphics[width=\linewidth]{img/inquinamentoDataCenter.png}
        \caption{{fonte \href{https://www.agendadigitale.eu/smart-city/il-digitale-non-e-un-pasto-gratis-quanto-inquinano-i-data-center-e-come-ridurne-limpatto/}{Agenda Digitale: Il digitale non è un pasto gratis}}}
    \end{figure}
\end{frame}

\begin{frame}{AI E DATA CENTER}
    \begin{figure}
        \includegraphics[width=.9\linewidth]{img/impattiAmbientaliAI.png}
        \caption{
            Immagine creata utilizzando screenshots tratti dai seguenti articoli:
            \href{https://attivissimo.blogspot.com/2024/08/anteprima-podcast-rsi-lia-ha-troppa.html}{Il Disinformatico},
            \href{https://www.ilpost.it/2026/01/09/intelligenza-artificiale-consumo-acqua/}{Il Post},
            \href{https://www.agendadigitale.eu/smart-city/la-sfida-dei-data-center-alimentare-lai-senza-distruggere-il-pianeta/}{Agenda Digitale},
            \href{https://www.geopop.it/intelligenza-artificiale-consumi-energia/}{GeoPop},
            \href{https://www.iea.org/}{IEA: International Energy Agency} 
        }
    \end{figure}
\end{frame}

\begin{frame}{ANATOMIA DI UN SISTEMA DI AI}
    \begin{figure}
        \href{https://anatomyof.ai/img/ai-anatomy-map.pdf}{\includegraphics[width=\linewidth]{img/anatomy.png}}
        \caption{{fonte: \href{https://anatomyof.ai/}{Kate Crawford: Anatomy of an AI System}}}
    \end{figure}
\end{frame}

\section{RISORSE, TERRE RARE E MATERIE PRIME CRITICHE}

\begin{frame}{SFRUTTAMENTO RISORSE: PALAQUIUM GUTTA}
    \begin{figure}
        \href{PalaquiumGutta.pdf}{\includegraphics[width=\linewidth]{img/guttaperca.png}}
        \caption{{fonte: \href{https://www.ibs.it/ne-intelligente-ne-artificiale-lato-libro-kate-crawford/e/9788815294197}{Kate Crawford: Né intelligente né artificiale.}}}
    \end{figure}
    \bigskip
    \tiny{\textbf{Approdondimento}}\\
    \tiny{\href{https://qz.com/785119/the-forgotten-tropical-tree-sap-that-set-off-a-victorian-tech-boom-and-gave-us-global-telecommunications}{Quartz: The story of the humble latex}}
\end{frame}

\begin{frame}{TERRE RARE REE (Rare Earth Elements)}
    \begin{figure}
        \includegraphics[width=\linewidth]{img/terrerare.png}
        \caption{{fonte: \href{https://www.geopop.it/terre-rare-cosa-sono-dove-si-estraggono-e-il-monopolio-cinese/}{Geopop: Terre rare, cosa sono e dove si estraggono}}}
    \end{figure}
\end{frame}

\begin{frame}{TERRE RARE REE (Rare Earth Elements)}
    \begin{figure}
        \includegraphics[width=.9\linewidth]{img/sfruttamentoRisorse.png}
        \caption{
            Immagine creata utilizzando screenshots tratti dai seguenti articoli:
            \href{https://www.editorialedomani.it/tecnologia/intelligenza-artificiale-quanto-inquina-miniere-litio-discariche-kinx6xvj}{Domani},
            \href{https://www.tomshw.it/smartphone/il-lago-tossico-dellhi-tech-il-prezzo-che-paga-la-cina-della-crescita-a-doppia-cifra/}{Tom's Hardware},
            \href{https://www.agendadigitale.eu/smart-city/terre-rare-lo-sporco-segreto-della-transizione-energetica-ecologica/}{Agenda Digitale},
            \href{https://www.polito.it/ateneo/comunicazione-e-ufficio-stampa/poliflash/le-terre-rare-sono-solo-la-punta-dell-iceberg}{Politecnico di Torino}
        }
    \end{figure}
\end{frame}

\begin{frame}{MATERIE PRIME CRITICHE}
    \begin{figure}
        \href{https://www.geopop.it/coltan-il-minerale-essenziale-per-gli-smartphone-estratto-da-minatori-sfruttati-in-congo/}{\includegraphics[width=\linewidth]{img/play.png}}
        \caption{{Fonte \href{https://www.geopop.it/coltan-il-minerale-essenziale-per-gli-smartphone-estratto-da-minatori-sfruttati-in-congo/}{Geopop: Coltan}}}
    \end{figure}
    \bigskip
    \tiny{\textbf{Approdondimento}}\\
    \tiny{\href{https://www.consilium.europa.eu/it/infographics/critical-raw-materials/}{Regolamento europeo sulle materie prime critiche}}
\end{frame}

\begin{frame}{RICICLO DISPOSITIVI ELETTRONICI}
    \begin{columns}
        \column{.5\textwidth}
            \begin{figure}
                \href{https://www.geopop.it/come-si-riciclano-le-batterie/}{\includegraphics[width=\linewidth]{img/play.png}}
                \caption{{Fonte \href{https://www.geopop.it/come-si-riciclano-le-batterie/}{Geopop: Come si riciclano le batterie}}}
            \end{figure}
        \column{.5\textwidth}
        \begin{figure}
            \href{https://www.geopop.it/la-piu-grande-discarica-di-rifiuti-elettronici-al-mondo-si-trova-in-ghana/}{\includegraphics[width=\linewidth]{img/play.png}}
            \caption{{Fonte \href{https://www.geopop.it/la-piu-grande-discarica-di-rifiuti-elettronici-al-mondo-si-trova-in-ghana/}{Geopop: La discarica di Agbogbloshie}}}
        \end{figure}
    \end{columns}
\bigskip
\tiny{\textbf{Approdondimento}}\\
\tiny{\href{https://www.europarl.europa.eu/topics/it/article/20201208STO93325/rifiuti-elettronici-nell-ue-dati-e-cifre-infografica}{Rifiuti elettronici nell'UE: dati e cifre}}
\end{frame}

\section{ANTROPOCENE}

\begin{frame}{ANTROPOCENE}
    \begin{alertblock}{DEFINIZIONE}
        \begin{minipage}{0.98\linewidth}
            \justifying
            \textbf{Antropocene} è un termine proposto per designare l'attuale epoca geologica, 
            nella quale \textbf{l'essere umano in quanto forza agente ha influenzato in modo così profondo 
            il pianeta e le sue forme di vita tanto da incidere sui processi geologici della Terra 
            attraverso le sue attività} (modifiche territoriali, strutturali e climatiche).\\
            \bigskip
            \tiny{\textbf{Approfondimento}}\\
            \tiny{\href{https://theanthropocene.org/}{The Anthropocene Project}}
        \end{minipage}
    \end{alertblock}
\end{frame}

\begin{frame}{ANTROPOCENE}
    \begin{figure}
        \href{https://earth.google.com/web/data=MkEKPwo9CiExa25uLXJXdmFrM0RmR29zRHM5Z1daVW1jUVBPSVR6N1kSFgoUMDM5QTJDQTc4RDEyQjQ1QzMxMDkgAUICCABKCAiH2oKrAhAB}{\includegraphics[width=\linewidth]{img/antropocene.png}}
        \caption{{fonte: \href{https://earth.google.com/web/data=MkEKPwo9CiExa25uLXJXdmFrM0RmR29zRHM5Z1daVW1jUVBPSVR6N1kSFgoUMDM5QTJDQTc4RDEyQjQ1QzMxMDkgAUICCABKCAiH2oKrAhAB}{Google Earth: The Anthropocene}}}
    \end{figure}
    \bigskip
    \tiny{\textbf{Film}}\\
    \tiny{\href{https://it.chili.com/content/antropocene-l-epoca-umana-2018/04eee8dc-face-48c4-9b51-bed21df91d80}{Antropocene: L'epoca umana}}
\end{frame}

\section{TOOLS ANALISI DATI}

\begin{frame}{TOOLS ANALISI DATI}
    \begin{alertblock}{DATI CLIMATICI}
        \begin{minipage}{0.98\linewidth}
            \justifying
            \begin{itemize}
                \item \href{https://www.iqair.com/it/world-air-quality}{\textbf{IQAir}}: Monitoraggio qualità 
                dell'aria in tempo reale a livello globale;
                \item \href{https://www.climatewatchdata.org/}{\textbf{Climate Watch}}: Dati e analisi sui cambiamenti climatici a livello globale;
                \item \href{https://www.climatiq.io/data}{\textbf{Climatiq}}: Il più grande database di fattori di emissione controllati.
            \end{itemize} 
        \end{minipage}
    \end{alertblock}
    \begin{alertblock}{DATI ENERGETICI}
        \begin{minipage}{0.98\linewidth}
            \justifying
            \begin{itemize}
                \item \href{https://www.iea.org}{\textbf{IEA}}: Informazioni e dati sulle energie e sulle 
                emissioni a livello globale.
            \end{itemize} 
        \end{minipage}
    \end{alertblock}
\end{frame}

\end{document}