% FONTE TEMA https://github.com/matze/mtheme
\documentclass[aspectratio=1610]{beamer}
%\documentclass[aspectratio=1610, handout]{beamer}
\usepackage[utf8]{inputenc}
\usepackage{ragged2e}
\usepackage{xcolor}
\usepackage[italian]{babel}
\usepackage{multirow}
\usepackage{silence}
\WarningFilter{beamer}{}
\WarningFilter{metropolis}{}
\usetheme[progressbar=frametitle,titleformat=smallcaps]{metropolis}
\setbeamertemplate{frame numbering}[fraction]
\setbeamercovered{dynamic}
\definecolor{rosso}{RGB}{255, 0, 0}
\definecolor{giallo}{RGB}{254,212,23}
\hypersetup{colorlinks=true,linkcolor=black,urlcolor=rosso}
\setbeamercolor{palette primary}{fg=black, bg=giallo}
\setbeamercolor{background canvas}{bg=white}
\setbeamercolor{normal text}{fg=black}
\setbeamercolor{progress bar}{fg=rosso}
\setbeamercolor{framesubtitle}{fg=rosso}
\setbeamercolor{normal text .dimmed}{fg=giallo}
\setbeamercolor{block title alerted}{fg=rosso, bg=giallo}
\setbeamerfont{caption}{size=\tiny}
\setbeamerfont{caption name}{size=\tiny}
\setlength{\abovecaptionskip}{0pt}
\makeatletter
\metroset{block=fill}
\setlength{\metropolis@progressinheadfoot@linewidth}{1pt} 
\setlength{\metropolis@progressonsectionpage@linewidth}{1pt}
\setlength{\metropolis@titleseparator@linewidth}{1pt}
\makeatother

\title{IMPATTI AMBIENTALI DELLE TECNOLOGIE DIGITALI}
\subtitle{Crisi climatica e sostenibilità}
\date{}
\institute{\textit{
        Fonti:
        \begin{itemize}
            \item[-] \href{https://www.ibs.it/ne-intelligente-ne-artificiale-lato-libro-kate-crawford/e/9788815294197}{Né intelligente né artificiale. Il lato oscuro dell'IA}
            \item[-] \href{www.nomefonte2.it}{NOME FONTE 2}
        \end{itemize}
    }
}

\begin{document}

\begin{frame}[plain, noframenumbering]
    \titlepage
\end{frame}

\section{CLOUD E DATA CENTER}

\begin{frame}{CLOUD E DATA CENTER}
    \begin{columns}
        \column{.5\textwidth}
            \begin{alertblock}{DEFINIZIONE}
                \begin{minipage}{0.96\linewidth}
                    \justifying
                    Il \textbf{cloud} consente agli utenti di accedere agli stessi file e alle stesse 
                    applicazioni da ogni dispositivo \textbf{tramite la rete Internet}, perché l'elaborazione e 
                    l'archiviazione hanno luogo in \textbf{server} che si trovano in \textbf{data center}, 
                    invece che localmente, nel dispositivo dell'utente.\\
                    \bigskip
                    \tiny{\textbf{Approfondimento}}\\
                    \tiny{\href{https://www.cloudflare.com/it-it/learning/cloud/what-is-the-cloud/}{Cos'è il cloud secondo Cloudflare}}
                \end{minipage}
            \end{alertblock}
        \column{.5\textwidth}
            \begin{figure}
                \includegraphics[width=.8\linewidth]{img/cloud_1.png}
                \caption{{creata con \href{www.chatgpt.com}{ChatGPT}}}
            \end{figure}
    \end{columns}
\end{frame}

\begin{frame}{CLOUD E DATA CENTER}
    \begin{columns}
        \column{.5\textwidth}
            \begin{alertblock}{DEFINIZIONE}
                \begin{minipage}{0.96\linewidth}
                    \justifying
                    \textit{``Il \textbf{cloud} è una tecnologia estrattiva ad \textbf{alta intensità di risorse} che 
                    \textbf{converte l'acqua e l'elettricità in potenza computazionale}, lasciando dietro di 
                    sé una quantità considerevole di \textbf{danni ambientali che poi cela alla vista}''.}\\
                    \bigskip
                    \tiny{\textbf{Citazione}}\\
                    \tiny{\href{https://www.tunghui.hu/books/a-prehistory-of-the-cloud}{Tung-Hui Hu: A Prehistory of the Cloud}}
                \end{minipage}
            \end{alertblock}
        \column{.5\textwidth}
            \begin{figure}
                \includegraphics[width=.8\linewidth]{img/cloud_2.png}
                \caption{{creata con \href{www.chatgpt.com}{ChatGPT}}}
            \end{figure}
    \end{columns}
\end{frame}

\begin{frame}{CLOUD E DATA CENTER}
    \begin{figure}
        \includegraphics[width=\linewidth]{img/digitalizzazione.png}
        \caption{{fonte \href{https://luissuniversitypress.it/inquinamento-cloud-articolo-andrea-daniele-signorelli-digitalizzazione-lmdp/}{Andrea Daniele Signorelli: Inquinamento cloud}}}
    \end{figure}
\end{frame}

\end{document}