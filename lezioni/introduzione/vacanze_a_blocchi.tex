% FONTE TEMA https://github.com/matze/mtheme
\documentclass[aspectratio=1610]{beamer}
%\documentclass[aspectratio=1610, handout]{beamer}
\usepackage[utf8]{inputenc}
\usepackage{ragged2e}
\usepackage{xcolor}
\usepackage[italian]{babel}
\usepackage{multirow}
\usepackage{silence}
\usepackage{tikz}
\usepackage[normalem]{ulem}
\WarningFilter{beamer}{}
\WarningFilter{metropolis}{}
\usetheme[progressbar=frametitle,titleformat=smallcaps]{metropolis}
\setbeamertemplate{frame numbering}[fraction]
\setbeamercovered{dynamic}
\definecolor{rosso}{RGB}{255, 0, 0}
\definecolor{giallo}{RGB}{254,212,23}
\hypersetup{colorlinks=true,linkcolor=black,urlcolor=rosso}
\setbeamercolor{palette primary}{fg=black, bg=giallo}
\setbeamercolor{background canvas}{bg=white}
\setbeamercolor{normal text}{fg=black}
\setbeamercolor{progress bar}{fg=rosso}
\setbeamercolor{framesubtitle}{fg=rosso}
\setbeamercolor{normal text .dimmed}{fg=giallo}
\setbeamercolor{block title alerted}{fg=rosso, bg=giallo}
\setbeamerfont{caption}{size=\tiny}
\setbeamerfont{caption name}{size=\tiny}
\setlength{\abovecaptionskip}{0pt}
\makeatletter
\metroset{block=fill}
\setlength{\metropolis@progressinheadfoot@linewidth}{1pt} 
\setlength{\metropolis@progressonsectionpage@linewidth}{1pt}
\setlength{\metropolis@titleseparator@linewidth}{1pt}
\makeatother

\title{VACANZE A BLOCCHI}
\subtitle{Gioco a quattro squadre}
\date{}
\institute{}

\begin{document}

\begin{frame}[plain, noframenumbering]
    \titlepage
\end{frame}

\begin{frame}{VACANZE A BLOCCHI - CREAZIONE DEI DATI (5 MINUTI) }
    \begin{itemize}
        \item Creare cinque bigliettini, uno per ciascuno dei seguenti temi:
              \begin{enumerate}
                  \item \textbf{Luogo} visitato durante le vacanze;
                  \item \textbf{Mezzo} di trasporto utilizzato durante le vacanze;
                  \item \textbf{Persona} con la quale si è andati in vacanza;
                  \item \textbf{Attività} svolta durante le vacanze;
                  \item \textbf{Oggetto} tipico utilizzato durante le vacanze.
              \end{enumerate}
        \pause
        \item Mettete i bigliettini sulla cattedra.
    \end{itemize}
\end{frame}

\begin{frame}{VACANZE A BLOCCHI - FORMAZIONE DELLE SQUADRE (5 MINUTI)}
    \begin{enumerate}
        \item Dividetevi in quattro squadre, posizionandovi nei quattro angoli dell'aula;
        \pause
        \item Scegliete un capitano;
        \pause
        \item Ogni capitano sceglie il colore della squadra.
    \end{enumerate}
\end{frame}

\begin{frame}{VACANZE A BLOCCHI - ASSEGNAZIONE DEI DATI (5 MINUTI)} 
    \begin{enumerate}
        \item Ogni capitano pesca cinque bigliettini;
        \pause
        \item Ogni squadra inventa un elemento \textbf{segreto} aggiuntivo a scelta tra i cinque temi precedenti: luogo, mezzo, persona, attività, oggetto;
        \pause
        \item Ogni squadra scrive su un foglio le cinque variabili assegnandone i valori pescati e l'elemento segreto:
                \begin{itemize}
                    \item[-] Luogo = X
                    \item[-] Mezzo = [Y,\textbf{Z}]
                    \item[-] \sout{Persona} =
                    \item[-] Attività =
                    \item[-] Oggetto =
                \end{itemize}
    \end{enumerate}
\end{frame}

\begin{frame}{VACANZE A BLOCCHI - CREAZIONE STORIA (15 MINUTI)}
    \begin{minipage}{0.98\linewidth}
        \centering
        \huge
        \textit{Ogni squadra scrive una storia sul foglio utilizzando tutte le variabili a sua disposizione, includendo 
        necessariamente un termine informatico utilizzato nel precedente anno scolastico.}\\
    \end{minipage}\\
\end{frame}

\begin{frame}{VACANZE A BLOCCHI - PRESENTAZIONE (3 MINUTI A SQUADRA)}
    \begin{itemize}
        \item A turno i capitani delle squadre trascrivono le variabili alla lavagna (senza rivelare quale sia l'elemento segreto);
        \pause
        \item A turno i capitani raccontano a voce la storia della propria squadra;
        \pause
        \item Ogni squadra avversaria scommette su quale sia l'elemento segreto della squadra che sta presentando.
    \end{itemize}
\end{frame}

\begin{frame}{VACANZE A BLOCCHI - PUNTEGGI}
    \begin{itemize}
        \item Le squadre avversarie devono dare un voto alla storia raccontata, da 1 a 10;
        \pause
        \item Ogni squadra che indovina l'elemento segreto della squadra che sta presentando guadagna 5 punti;
        \pause
        \item La squadra che ha raccontato la storia guadagna 2 punti per ogni squadra che non ha indovinato l'elemento segreto;
        \pause
        \item 5 punti bonus alla squadra che include nella propria storia il nome esatto di un luogo specifico nel quale è stato in vacanza il Prof.
    \end{itemize}
\end{frame}
\end{document}