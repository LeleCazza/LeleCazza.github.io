% FONTE TEMA https://github.com/matze/mtheme
\documentclass[aspectratio=1610]{beamer}
%\documentclass[aspectratio=1610, handout]{beamer}
\usepackage[utf8]{inputenc}
\usepackage{ragged2e}
\usepackage{xcolor}
\usepackage[italian]{babel}
\usepackage{multirow}
\usepackage{silence}
\usepackage{tikz}
\WarningFilter{beamer}{}
\WarningFilter{metropolis}{}
\usetheme[progressbar=frametitle,titleformat=smallcaps]{metropolis}
\setbeamertemplate{frame numbering}[fraction]
\setbeamercovered{dynamic}
\definecolor{rosso}{RGB}{255, 0, 0}
\definecolor{giallo}{RGB}{254,212,23}
\hypersetup{colorlinks=true,linkcolor=black,urlcolor=rosso}
\setbeamercolor{palette primary}{fg=black, bg=giallo}
\setbeamercolor{background canvas}{bg=white}
\setbeamercolor{normal text}{fg=black}
\setbeamercolor{progress bar}{fg=rosso}
\setbeamercolor{framesubtitle}{fg=rosso}
\setbeamercolor{normal text .dimmed}{fg=giallo}
\setbeamercolor{block title alerted}{fg=rosso, bg=giallo}
\setbeamerfont{caption}{size=\tiny}
\setbeamerfont{caption name}{size=\tiny}
\setlength{\abovecaptionskip}{0pt}
\makeatletter
\metroset{block=fill}
\setlength{\metropolis@progressinheadfoot@linewidth}{1pt} 
\setlength{\metropolis@progressonsectionpage@linewidth}{1pt}
\setlength{\metropolis@titleseparator@linewidth}{1pt}
\makeatother

\title{INTRODUZIONE}
\subtitle{Informatica}
\date{}
\institute{}

\begin{document}

\begin{frame}[plain, noframenumbering]
    \titlepage
\end{frame}

\begin{frame}{CHE PAROLE ASSOCI AL TERMINE INFORMATICA?}
\end{frame}

\section{REGOLE}

\begin{frame}{REGOLE DELLA CLASSE}
    \begin{enumerate}
        \item Definire una regola;
        \pause
        \item Proporre tre diverse conseguenze in caso di mancato rispetto della regola;
        \pause
        \item Associare un identificativo univoco per ogni conseguenza;
        \pause
        \item Votare la conseguenza preferita scrivendone l'identificativo su un bigliettino di carta;
        \pause
        \item Effettuare lo spoglio e determinare la conseguenza più votata;
        \pause
        \item Creare due regole.
    \end{enumerate}
\end{frame}

\begin{frame}{CONGRATULAZIONI!}
    \begin{minipage}{0.98\linewidth}
        \centering
        \huge
        \textit{``QUESTA \'E INFORMATICA!''}\\
    \end{minipage}\\
    \bigskip
    \tiny{\textbf{Definizione}}\\
    \tiny{\href{https://it.wikipedia.org/wiki/Informatica}{definizione di informatica}}
\end{frame}

\begin{frame}{PERCH\'E \'E INFORMATICA?!}
    \begin{enumerate}
        \item Definire una regola;
        \item Proporre tre diverse conseguenze in caso di mancato rispetto della regola;
        \item Associare un identificativo univoco per ogni conseguenza;
        \item Votare la conseguenza preferita scrivendone l'identificativo su un bigliettino di carta;
        \item Effettuare lo spoglio e determinare la conseguenza più votata;
        \item Creare due regole.
    \end{enumerate}
\end{frame}

\begin{frame}{PERCH\'E \'E INFORMATICA?!}
    \begin{enumerate}
        \item Definire una regola;
        \item Proporre tre diverse conseguenze in caso di mancato rispetto della regola;
        \item Associare un identificativo univoco per ogni conseguenza;
        \item Votare la conseguenza preferita scrivendone l'identificativo su un bigliettino di carta;
        \item Effettuare lo spoglio e determinare la conseguenza più votata;
        \item Creare due regole.
    \end{enumerate}
    \begin{tikzpicture}[remember picture, overlay]
        \draw[line width=2pt, color=red] (3.5,3) circle (4cm);
                \node[xshift=10cm,yshift=6cm,color=red]{\textbf{ESECUZIONE DI UN ``ALGORITMO''}};
    \end{tikzpicture}
\end{frame}

\begin{frame}{PERCH\'E \'E INFORMATICA?!}
    \begin{enumerate}
        \item Definire una \textcolor{red}{regola};
        \item Proporre tre diverse conseguenze in caso di mancato rispetto della regola;
        \item Associare un identificativo univoco per ogni conseguenza;
        \item Votare la conseguenza preferita scrivendone l'identificativo su un bigliettino di carta;
        \item Effettuare lo spoglio e determinare la conseguenza più votata;
        \item Creare due regole.
    \end{enumerate}
    \begin{tikzpicture}[remember picture, overlay]
        \draw[line width=2pt, color=red] (3.4,4.8) circle (1.3cm);
                \node[xshift=9.2cm,yshift=5.5cm,color=red]{\textbf{CREAZIONE DI ``REGOLE'' E ``DEFINIZIONI''}};
    \end{tikzpicture}
\end{frame}

\begin{frame}{PERCH\'E \'E INFORMATICA?!}
    \begin{enumerate}
        \item Definire una regola;
        \item Proporre tre diverse \textcolor{red}{conseguenze} in caso di mancato rispetto della regola;
        \item Associare un identificativo univoco per ogni conseguenza;
        \item Votare la conseguenza preferita scrivendone l'identificativo su un bigliettino di carta;
        \item Effettuare lo spoglio e determinare la conseguenza più votata;
        \item Creare due regole.
    \end{enumerate}
    \begin{tikzpicture}[remember picture, overlay]
        \draw[line width=2pt, color=red] (5.1,4.1) circle (1.3cm);
                \node[xshift=8.2cm,yshift=5.8cm,color=red]{\textbf{CREAZIONE DI ``SELEZIONI LOGICHE'': CAUSA - EFFETTO}};
    \end{tikzpicture}
\end{frame}

\begin{frame}{PERCH\'E \'E INFORMATICA?!}
    \begin{enumerate}
        \item Definire una regola;
        \item Proporre tre diverse conseguenze in caso di mancato rispetto della regola;
        \item Associare un \textcolor{red}{identificativo} univoco per ogni conseguenza;
        \item Votare la conseguenza preferita scrivendone l'identificativo su un bigliettino di carta;
        \item Effettuare lo spoglio e determinare la conseguenza più votata;
        \item Creare due regole.
    \end{enumerate}
    \begin{tikzpicture}[remember picture, overlay]
        \draw[line width=2pt, color=red] (3.9,3.5) circle (1.3cm);
                \node[xshift=8.4cm,yshift=4.8cm,color=red]{\textbf{CREAZIONE DI UNA ``CODIFICA''}};
    \end{tikzpicture}
\end{frame}

\begin{frame}{PERCH\'E \'E INFORMATICA?!}
    \begin{enumerate}
        \item Definire una regola;
        \item Proporre tre diverse conseguenze in caso di mancato rispetto della regola;
        \item Associare un identificativo univoco per ogni conseguenza;
        \item Votare la conseguenza preferita scrivendone l'identificativo su un \textcolor{red}{bigliettino} di carta;
        \item Effettuare lo spoglio e determinare la conseguenza più votata;
        \item Creare due regole.
    \end{enumerate}
    \begin{tikzpicture}[remember picture, overlay]
        \draw[line width=2pt, color=red] (12.3,2.9) circle (1.3cm);
                \node[xshift=9.8cm,yshift=1.1cm,color=red]{\textbf{UTILIZZO DI UN ``MEZZO DI COMUNICAZIONE''}};
    \end{tikzpicture}
\end{frame}

\begin{frame}{PERCH\'E \'E INFORMATICA?!}
    \begin{enumerate}
        \item Definire una regola;
        \item Proporre tre diverse conseguenze in caso di mancato rispetto della regola;
        \item Associare un identificativo univoco per ogni conseguenza;
        \item Votare la conseguenza preferita scrivendone l'identificativo su un bigliettino di carta;
        \item Effettuare lo spoglio e \textcolor{red}{determinare} la conseguenza più votata;
        \item Creare due regole.
    \end{enumerate}
    \begin{tikzpicture}[remember picture, overlay]
        \draw[line width=2pt, color=red] (5.3,1.7) circle (1.3cm);
                \node[xshift=10.3cm,yshift=1cm,color=red]{\textbf{``DECODIFICA'' DELLE INFORMAZIONI}};
    \end{tikzpicture}
\end{frame}

\begin{frame}{CHE PAROLE ASSOCI AL TERMINE INFORMATICA?}
    \begin{itemize}
        \item Dati, informazioni, definizioni e simboli formali;
        \item Rappresentazione, interpretazione e memorizzazione dei dati: codifica e decodifica;
        \item Algoritmi, logica, programmazione e risoluzione di problemi;
        \item Comunicazione e mezzi di trasmissione dati;
        \item \textbf{Elaborazione automatica dell'informazione}.
    \end{itemize}
\end{frame}

\end{document}