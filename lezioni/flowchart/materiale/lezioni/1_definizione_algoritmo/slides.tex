% FONTE TEMA https://github.com/matze/mtheme
\documentclass[aspectratio=1610]{beamer}
%\documentclass[aspectratio=1610, handout]{beamer}
\usepackage[utf8]{inputenc}
\usepackage{ragged2e}
\usepackage{xcolor}
\usepackage[italian]{babel}
\usepackage{multirow}
\usepackage{silence}
\WarningFilter{beamer}{}
\WarningFilter{metropolis}{}
\usetheme[progressbar=frametitle,titleformat=smallcaps]{metropolis}
\setbeamertemplate{frame numbering}[fraction]
\setbeamercovered{dynamic}
\definecolor{rosso}{RGB}{255, 0, 0}
\definecolor{giallo}{RGB}{254,212,23}
\hypersetup{colorlinks=true,linkcolor=black,urlcolor=rosso}
\setbeamercolor{palette primary}{fg=black, bg=giallo}
\setbeamercolor{background canvas}{bg=white}
\setbeamercolor{normal text}{fg=black}
\setbeamercolor{progress bar}{fg=rosso}
\setbeamercolor{framesubtitle}{fg=rosso}
\setbeamercolor{normal text .dimmed}{fg=giallo}
\setbeamercolor{block title alerted}{fg=rosso, bg=giallo}
\setbeamerfont{caption}{size=\tiny}
\setbeamerfont{caption name}{size=\tiny}
\setlength{\abovecaptionskip}{0pt}
\makeatletter
\metroset{block=fill}
\setlength{\metropolis@progressinheadfoot@linewidth}{1pt} 
\setlength{\metropolis@progressonsectionpage@linewidth}{1pt}
\setlength{\metropolis@titleseparator@linewidth}{1pt}
\makeatother

\title{DEFINIZIONE DI ALGORITMO}
\subtitle{Che cos'è un algoritmo?}
\date{}
\institute{\textit{
        Fonti:
        \begin{itemize}
            \item[-] \href{https://it.wikipedia.org/wiki/The_Art_of_Computer_Programming}{The Art of Computer Programming - Donald E. Knuth}
        \end{itemize}
    }
}

\begin{document}

\begin{frame}[plain, noframenumbering]
    \titlepage
\end{frame}

\begin{frame}{DEFINIZIONE DI ALGORITMO}
    \begin{alertblock}{DEFINIZIONE}
        \begin{minipage}{0.98\linewidth}
            \justifying
            Un algoritmo è un insieme finito di regole che determinano una 
            sequenza di passi per risolvere uno specifico tipo di problemi o per 
            svolgere uno specifico tipo di compiti.\\
            \bigskip
            \pause
            Per essere definito tale, un algoritmo deve inoltre soddisfare quattro 
            proprietà fondamentali:
            \begin{itemize}
                \item Finitezza;
                \item Precisione;
                \item Dati in ingresso e in uscita;
                \item Fattibilità.
            \end{itemize}
            \bigskip
        \end{minipage}
    \end{alertblock}
\end{frame}

\begin{frame}{DEFINIZIONE DI ALGORITMO}
    \begin{alertblock}{FINITEZZA}
        \begin{minipage}{0.98\linewidth}
            \justifying
            Un algoritmo deve terminare dopo un numero finito di passi. Una procedura 
            che non termina non andrebbe chiamata algoritmo, ma metodo computazionale. 
            \bigskip
        \end{minipage}
    \end{alertblock}
\end{frame}

\begin{frame}{DEFINIZIONE DI ALGORITMO}
    \begin{alertblock}{PRECISIONE}
        \begin{minipage}{0.98\linewidth}
            \justifying
            Ogni passo di un algoritmo deve essere definito precisamente; le azioni da 
            eseguire devono essere specificate rigorosamente e in maniera non ambigua 
            per ogni caso possibile.
            \bigskip
        \end{minipage}
    \end{alertblock}
\end{frame}

\begin{frame}{DEFINIZIONE DI ALGORITMO}
    \begin{alertblock}{DATI IN INGRESSO E IN USCITA}
        \begin{minipage}{0.98\linewidth}
            \justifying
            Un algoritmo ha zero o più input presi da un insieme specificato di oggetti.\\
            Un algoritmo ha uno o più output, ovvero risultati che stanno in una relazione 
            specificata con l’input. Dati certi input, l’algoritmo produce sempre gli stessi output.
            Quando l’algoritmo non ha nessun dato in ingresso, l’output prodotto è sempre lo stesso.
            \bigskip
        \end{minipage}
    \end{alertblock}
\end{frame}

\begin{frame}{DEFINIZIONE DI ALGORITMO}
    \begin{alertblock}{FATTIBILIT\'A}
        \begin{minipage}{0.98\linewidth}
            \justifying
            Un algoritmo per poter essere tale deve essere effettivamente eseguibile, nel senso che
            tutte le sue operazioni devono essere sufficientemente elementari da poter
            essere svolte in maniera esatta e in un periodo di tempo finito.
            \bigskip
        \end{minipage}
    \end{alertblock}
\end{frame}

\end{document}