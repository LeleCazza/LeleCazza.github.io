\documentclass[aspectratio=1610]{beamer}
\usepackage[utf8]{inputenc}
\usepackage{ragged2e}
\usepackage{xcolor}
\usepackage[italian]{babel}
\usetheme[progressbar=frametitle,titleformat=smallcaps]{metropolis}
\setbeamertemplate{frame numbering}[fraction]
\setbeamercovered{dynamic}
\definecolor{rosso}{RGB}{255, 0, 0}
\definecolor{giallo}{RGB}{254,212,23}
\hypersetup{colorlinks=true,linkcolor=black,urlcolor=rosso}
\setbeamercolor{palette primary}{fg=black, bg=giallo}
\setbeamercolor{background canvas}{bg=white}
\setbeamercolor{normal text}{fg=black}
\setbeamercolor{progress bar}{fg=rosso}
\setbeamercolor{framesubtitle}{fg=rosso}
\setbeamercolor{normal text .dimmed}{fg=giallo}
\setbeamercolor{block title alerted}{fg=rosso, bg=giallo}
\setbeamerfont{caption}{size=\tiny}
\setbeamerfont{caption name}{size=\tiny}
\setlength{\abovecaptionskip}{0pt}
\makeatletter
\metroset{block=fill}
\setlength{\metropolis@progressinheadfoot@linewidth}{1pt} 
\setlength{\metropolis@progressonsectionpage@linewidth}{1pt}
\setlength{\metropolis@titleseparator@linewidth}{1pt}
\makeatother

\title{LEGO SPIKE PRIME}
\subtitle{Creazione di un robot e sponsorizzazione tramite Google Sites}
\date{}

\begin{document}

\begin{frame}[plain, noframenumbering]
    \titlepage
\end{frame}

\section{SEZIONE 1: INDICE DEL PROGETTO}

\begin{frame}{INDICE DEL PROGETTO}
    \begin{itemize}
        \item \textbf{CREAZIONE DEL ROBOT}
        \pause
        \item \textbf{CREAZIONE DEL MATERIALE PER LA SPONSORIZZAZIONE}
        \pause
        \item \textbf{CREAZIONE DEL SITO WEB E SPONSORIZZAZIONE}
    \end{itemize}
\end{frame}

\begin{frame}{CREAZIONE DEL ROBOT}
    \begin{itemize}
        \item \textbf{SCELTA DEL ROBOT}: costruzione di un robot partendo dai modelli base
        presenti sul sito \href{https://spike.legoeducation.com/prime/models/}{legoeducation};
        \pause
        \item \textbf{PERSONALIZZAZIONE}: personalizzazione del robot aggiungendo modifiche stilistiche 
        o funzionali;
        \pause
        \item \textbf{PROGRAMMAZIONE}: programmazione del robot per lo svolgimento delle sue funzioni.
    \end{itemize}
\end{frame}

\begin{frame}{CREAZIONE DEL MATERIALE PER LA SPONSORIZZAZIONE}
    \begin{itemize}
        \item \textbf{CREAZIONE DI CONTENUTI MULTIMEDIALI}: raccolta di foto, video, e altro materiale 
        multimediale per la sponsorizzazione del robot; 
        \pause
        \item \textbf{CREAZIONE DELL'AZIENDA}: creazione e progettazione dell'azienda (logo, identità, sedi, bilancio, ecc.) 
        produttrice del robot creato.
    \end{itemize}
\end{frame}

\begin{frame}{CREAZIONE DEL SITO WEB E SPONSORIZZAZIONE}
    \begin{itemize}
        \item \textbf{PROGETTAZIONE DEL SITO}: progettazione dell'organizzazione del sito web 
        (numero di pagine previste, organizzazione dei contenuti, homepage, ecc.);
        \pause
        \item \textbf{CREAZIONE DEL SITO}: creazione effettiva del sito web tramite \href{https://sites.google.com/}{Google Sites};
        \pause
        \item \textbf{PRESENTAZIONE}: Realizzazione tramite \href{https://www.canva.com/}{Canva} di una presentazione 
        per l'esposizione del proprio progetto. 
    \end{itemize}
\end{frame}

\end{document}