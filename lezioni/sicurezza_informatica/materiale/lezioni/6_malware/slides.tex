\documentclass[aspectratio=1610]{beamer}
%\documentclass[aspectratio=1610, handout]{beamer}
\usepackage[utf8]{inputenc}
\usepackage{ragged2e}
\usepackage{xcolor}
\usepackage[italian]{babel}
\usepackage{multirow}
\usetheme[progressbar=frametitle,titleformat=smallcaps]{metropolis}
\setbeamertemplate{frame numbering}[fraction]
\setbeamercovered{dynamic}
\definecolor{rosso}{RGB}{255, 0, 0}
\definecolor{giallo}{RGB}{254,212,23}
\hypersetup{colorlinks=true,linkcolor=black,urlcolor=rosso}
\setbeamercolor{palette primary}{fg=black, bg=giallo}
\setbeamercolor{background canvas}{bg=white}
\setbeamercolor{normal text}{fg=black}
\setbeamercolor{progress bar}{fg=rosso}
\setbeamercolor{framesubtitle}{fg=rosso}
\setbeamercolor{normal text .dimmed}{fg=giallo}
\setbeamercolor{block title alerted}{fg=rosso, bg=giallo}
\setbeamerfont{caption}{size=\tiny}
\setbeamerfont{caption name}{size=\tiny}
\setlength{\abovecaptionskip}{0pt}
\makeatletter
\metroset{block=fill}
\setlength{\metropolis@progressinheadfoot@linewidth}{1pt} 
\setlength{\metropolis@progressonsectionpage@linewidth}{1pt}
\setlength{\metropolis@titleseparator@linewidth}{1pt}
\makeatother

\title{MALWARE}
\subtitle{Malware più diffusi}
\date{}
\institute{\textit{
        Fonti:
        \begin{itemize}
            \item[-] \href{https://it.wikipedia.org/wiki/Malware}{Wikipedia: Malware}
            \item[-] \href{https://www.fastweb.it/fastweb-plus/digital-magazine/cosa-sono-gli-attacchi-zero-click-e-come-difendersi?fwp_login_type=3}{Fastweb Plus}
            \item[-] \href{https://www.acn.gov.it/portale/documents/20119/728444/ACN_Ransomware_2024_CLEAR.pdf/76f76c2e-4c5e-e3cf-e76c-9f8032eea749?t=1734622549825}{Agenzia per la Cybersicurezza Nazionale}
            \item[-] \href{https://it.wikipedia.org/wiki/Virus_(informatica)}{Wikipedia: Virus}
            \item[-] \href{https://iris.polito.it/retrieve/handle/11583/2517684/60956/Mezzalama_Lioy_Metwalley_Anatomia_del_malware.pdf}{Politecnico di Torino: Anatomia del Malware}
            \item[-] \href{https://www.kaspersky.it/resource-center/threats/types-of-malware}{Kaspersky: Tipi di Malware}
        \end{itemize}
    }
}

\begin{document}

\begin{frame}[plain, noframenumbering]
    \titlepage
\end{frame}

\section{PREMESSA}

\begin{frame}{ATTENZIONE}
    \begin{alertblock}{ATTENZIONE}
        \begin{minipage}{0.98\linewidth}
            \justifying
            Le seguenti slide contengono materiale potenzialmente pericoloso, fornito \textbf{esclusivamente a 
            scopo didattico} per l'apprendimento delle tecniche di sicurezza informatica. Questi strumenti 
            devono essere utilizzati \textbf{in modo etico e responsabile}, esclusivamente per scopi legittimi come 
            il miglioramento della sicurezza e la protezione dei sistemi.\\
            \textbf{È vietato} utilizzare queste informazioni per attività \textbf{malevoli o illegali}. 
            Ogni uso improprio che violi le leggi o i principi etici è severamente sanzionato dalla legge.
        \end{minipage}
    \end{alertblock}
\end{frame}

\section{MALWARE}

\begin{frame}{MALWARE}
    \begin{alertblock}{DEFINIZIONE}
        \begin{minipage}{0.98\linewidth}
            \justifying
            Un \textbf{Malware} (abbreviazione dell'inglese malicious software, 
            letteralmente ``software malevolo''), indica un \textbf{qualsiasi programma informatico usato 
            per disturbare le operazioni svolte da un utente di un computer}. Termine coniato nel 1990, 
            precedentemente veniva chiamato virus per computer. \\
            Il malware non necessariamente è creato per arrecare danni tangibili ad un computer o un 
            sistema informatico, ma va inteso anche come un programma che può rubare di nascosto 
            informazioni di vario tipo, da commerciali a private, in genere senza essere rilevato dall'utente 
            anche per lunghi periodi di tempo.\\
            \bigskip
            \tiny{\textbf{Curiosità}}\\
            \tiny{\href{https://www.ibm.com/it-it/think/topics/malware-history}{Breve storia dei Malware}}
        \end{minipage}
    \end{alertblock}
\end{frame}

\section{RABBIT}

\begin{frame}{RABBIT}
    \begin{alertblock}{DEFINIZIONE}
        \begin{minipage}{0.98\linewidth}
            \justifying
            Tipo di malware che \textbf{attacca le risorse del sistema} duplicando in continuazione la propria 
            immagine su disco, o attivando nuovi processi a partire dal proprio eseguibile, 
            in modo da \textbf{consumare tutte le risorse disponibili} sul sistema in pochissimo tempo. 
            Il nome si riferisce proprio alla prolificità di questo "infestante".\\
            \bigskip
            \tiny{\textbf{Esempio}}\\
            \tiny{\href{https://it.wikipedia.org/wiki/Fork_bomb}{Fork Bomb}}
        \end{minipage}
    \end{alertblock}
\end{frame}

\begin{frame}{RABBIT - ESEMPIO FORK BOMB}
    \begin{enumerate}
        \item Creare un file di testo RABBIT.txt;
        \pause
        \item Inserire la Fork Bomb nel file digitando: \textbf{start cmd /k echo BOMB \textbf{$\mid$} \%0}
        \pause
        \item Modificare l’estensione del file da .txt a .bat
        \pause
        \item Preparare un terminale con l'istruzione: \textbf{taskkill /F /IM cmd.exe} per stoppare il malware;
        \pause
        \item Eseguire il file RABBIT.bat e terminarlo con il comando preparato sul terminale.
    \end{enumerate}                        
\end{frame}

\section{VIRUS}

\begin{frame}{VIRUS}
    \begin{columns}
        \column{.5\textwidth}
            \begin{alertblock}{DEFINIZIONE}
                \begin{minipage}{0.96\linewidth}
                    \justifying
                    Un \textbf{Virus} è un Malware che \textbf{infetta dei file in modo da creare copie di se stesso}, 
                    generalmente senza farsi rilevare dall'utente. Il termine viene usato per un programma che 
                    si integra in qualche codice eseguibile (incluso il sistema operativo) del sistema informatico 
                    vittima, in modo tale da \textbf{diffondersi su altro codice eseguibile} quando viene eseguito il 
                    codice che lo ospita, senza che l'utente ne sia a conoscenza.\\
                    \bigskip
                    \tiny{\textbf{Approfondimento}}\\
                    \tiny{\href{https://www.geopop.it/malware-i-virus-informatici-che-entrano-nei-nostri-computer-per-rubare-dati-come-possiamo-proteggerci/}{Geopop: Malware}}
                \end{minipage}
            \end{alertblock}
        \column{.5\textwidth}
            \begin{figure}
                \includegraphics[width=\linewidth]{img/virus.jpg}
                \caption{{fonte \href{https://iris.polito.it/retrieve/handle/11583/2517684/60956/Mezzalama_Lioy_Metwalley_Anatomia_del_malware.pdf}{Politecnico di Torino}}}
            \end{figure}
    \end{columns}
\end{frame}

\section{ELENCO PRINCIPALI MALWARE}

\begin{frame}{ELENCO PRINCIPALI MALWARE}
    \begin{alertblock}{PRINCIPALI MALWARE}
        \begin{minipage}{0.98\linewidth}
            \justifying
            Di seguito un elenco delle principali tipologie di Malware esistenti:
            \bigskip
            \begin{itemize}
                \pause
                \item \textbf{Adware}: Malware che mostra pubblicità indesiderata sul web o sul dispositivo;
                \pause
                \item \textbf{Spyware}: Malware che raccoglie informazioni sull'utente senza il suo consenso;
                \pause
                \item \textbf{Keylogger}: Spyware che registra le digitazioni da tastiera per rubare informazioni sensibili;
                \pause
                \item \textbf{Trojan}: Malware che si maschera da software legittimo per ingannare l'utente e infettare il sistema;
                \pause
                \item \textbf{Backdoor}: Trojan che crea un accesso nascosto al sistema per consentire l'accesso non autorizzato;
                \pause
                \item \textbf{Worm}: Virus che si replica autonomamente diffondendosi attraverso la rete;
            \end{itemize}
            \bigskip
            \tiny{\textbf{Curiosità}}\\
            \tiny{\href{https://learn.microsoft.com/it-it/unified-secops/malware-naming}{Microsoft: Nomenclatura Malware}}
        \end{minipage}
    \end{alertblock}
\end{frame}

\section{RANSOMWARE}

\begin{frame}{RANSOMWARE}
    \begin{alertblock}{DEFINIZIONE}
        \begin{minipage}{0.98\linewidth}
            \justifying
            Il ransomware è una tipologia di minaccia che ha lo \textbf{scopo di cifrare i dati del bene informatico 
            target in modo da comprometterne la disponibilità, integrità e riservatezza}. Inoltre, in questa 
            tipologia di minaccia spesso l’attaccante crea dei file, detti ransom notes, tramite i quali viene 
            richiesto alla vittima un riscatto in cambio dell’accesso ai propri dati. In alcuni casi i dati, prima 
            di essere cifrati, vengono \href{https://www.ibm.com/it-it/think/topics/data-exfiltration}{\textbf{esfiltrati}} in modo da offrire all’attaccante uno strumento in più di ricatto
            nei confronti della vittima.\\
            \bigskip
            \tiny{\textbf{Curiosità}}\\
            \tiny{\href{https://www.ransomfeed.it/}{Monitoraggio gruppi Ransomware}}
        \end{minipage}
    \end{alertblock}
\end{frame}

\section{ZERO CLICK}

\begin{frame}{ZERO CLICK}
    \begin{alertblock}{DEFINIZIONE}
        \begin{minipage}{0.98\linewidth}
            \justifying
            Questo tipo di malware \textbf{non richiede che la vittima esegua azioni}, ma sfrutta le 
            vulnerabilità dei software o la presenza di bug non ancora corretti per farsi strada 
            nei dispositivi e infettarli. Questo significa che \textbf{la vittima non si rende conto dei 
            malware} che operano sul proprio PC, smartphone o tablet e non ha modo di individuare 
            la minaccia alla sua sicurezza online.\\
            \bigskip
            \tiny{\textbf{Curiosità}}\\
            \tiny{\href{https://www.cybersecurity360.it/cybersecurity-nazionale/graphite-di-paragon-risposte/}{Graphite: Paragon Software Group}}
        \end{minipage}
    \end{alertblock}
\end{frame}

\section{ELENCO MALWARE STORICI}

\begin{frame}{ELENCO MALWARE STORICI}
    \begin{alertblock}{MALWARE STORICI}
        \begin{minipage}{0.98\linewidth}
            \justifying
            Di seguito un elenco di alcuni dei malware più famosi e pericolosi della storia dell'informatica ordinati 
            per anno di comparsa:
            \begin{itemize}
                \justifying
                \item 1971 - \href{https://www.geopop.it/creeper-e-il-primo-malware-della-storia-chi-lha-sviluppato-e-perche/}{\textbf{Creeper}}: Primo Malware (Worm) della storia;
                \item 2000 - \href{https://it.wikipedia.org/wiki/ILOVEYOU}{\textbf{Iloveyou}}: Malware (Worm) che sfruttava tecniche di social engineering via email (Phishing);
                \item 2010 - \href{https://it.wikipedia.org/wiki/Stuxnet}{\textbf{Stuxnet}}: Malware (Virus) sviluppato per sabotare il programma nucleare iraniano;
                \item 2016 - \href{https://www.cybersecurity360.it/nuove-minacce/pegasus-continua-a-diffondersi-i-motivi-e-le-best-practice-per-difendersi/}{\textbf{Pegasus}}: Malware (Spyware) sviluppato dalla NSO Group per spiare dispositivi mobili;
                \item 2017 - \href{https://www.cloudflare.com/it-it/learning/security/ransomware/wannacry-ransomware/}{\textbf{WannaCry}}: Malware (Ransomware) che si è diffuso in tutto il mondo sfruttando l'exploit ``EternalBlue'';
                \item 2024 - \href{https://www.cybersecurity360.it/cybersecurity-nazionale/graphite-di-paragon-risposte/}{\textbf{Graphite}}: Malware (Spyware) sviluppato da Paragon Software Group per spiare dispositivi mobili.
            \end{itemize}
        \end{minipage}
    \end{alertblock}
\end{frame}

\begin{frame}{Prof. Eugene Howard Spafford (Spaf)}
    \begin{minipage}{0.98\linewidth}
        \centering
        \huge
        \textit{``L’unico vero sistema sicuro è un sistema spento, chiuso in una gettata di cemento, 
        sigillato in una stanza rivestita di piombo protetta da guardie armate. 
        Ma anche in questo caso ho i miei dubbi.''}
    \end{minipage}
\end{frame}
\end{document}