\documentclass[aspectratio=1610]{beamer}
%\documentclass[aspectratio=1610, handout]{beamer}
\usepackage[utf8]{inputenc}
\usepackage{ragged2e}
\usepackage{xcolor}
\usepackage[italian]{babel}
\usepackage{multirow}
\usetheme[progressbar=frametitle,titleformat=smallcaps]{metropolis}
\setbeamertemplate{frame numbering}[fraction]
\setbeamercovered{dynamic}
\definecolor{rosso}{RGB}{255, 0, 0}
\definecolor{giallo}{RGB}{254,212,23}
\hypersetup{colorlinks=true,linkcolor=black,urlcolor=rosso}
\setbeamercolor{palette primary}{fg=black, bg=giallo}
\setbeamercolor{background canvas}{bg=white}
\setbeamercolor{normal text}{fg=black}
\setbeamercolor{progress bar}{fg=rosso}
\setbeamercolor{framesubtitle}{fg=rosso}
\setbeamercolor{normal text .dimmed}{fg=giallo}
\setbeamercolor{block title alerted}{fg=rosso, bg=giallo}
\setbeamerfont{caption}{size=\tiny}
\setbeamerfont{caption name}{size=\tiny}
\setlength{\abovecaptionskip}{0pt}
\makeatletter
\metroset{block=fill}
\setlength{\metropolis@progressinheadfoot@linewidth}{1pt} 
\setlength{\metropolis@progressonsectionpage@linewidth}{1pt}
\setlength{\metropolis@titleseparator@linewidth}{1pt}
\makeatother

\title{MALWARE \& ATTACKS}
\subtitle{Attacchi informatici e Malware più diffusi}
\date{}
\institute{\textit{
        Fonti:
        \begin{itemize}
            \item[-] \href{https://it.wikipedia.org/wiki/Attacco\_informatico}{Wikipedia}
        \end{itemize}
    }
}

\begin{document}

\begin{frame}[plain, noframenumbering]
    \titlepage
\end{frame}

\section{PREMESSA}

\begin{frame}{ATTENZIONE}
    \begin{alertblock}{ATTENZIONE}
        \begin{minipage}{0.98\linewidth}
            \justifying
            Le seguenti slide contengono materiale potenzialmente pericoloso, fornito \textbf{esclusivamente a 
            scopo didattico} per l'apprendimento delle tecniche di sicurezza informatica. Questi strumenti 
            devono essere utilizzati \textbf{in modo etico e responsabile}, esclusivamente per scopi legittimi come 
            il miglioramento della sicurezza e la protezione dei sistemi.\\
            \textbf{È vietato} utilizzare queste informazioni per attività \textbf{malevoli o illegali}. 
            Ogni uso improprio che violi le leggi o i principi etici è severamente sanzionato dalla legge.
        \end{minipage}
    \end{alertblock}
\end{frame}

\section{ATTACCHI INFORMATICI}

\begin{frame}{SNIFFIG DI RETE}
    \begin{alertblock}{DEFINIZIONE}
        \begin{minipage}{0.98\linewidth}
            \justifying
            Attività di \textbf{intercettazione passiva dei dati} che transitano in una rete telematica: 
            può essere svolta sia per scopi legittimi (ad esempio l'analisi e l'individuazione di 
            problemi di comunicazione o di tentativi di intrusione) sia per scopi illeciti contro 
            la sicurezza informatica (intercettazione fraudolenta di password o altre informazioni sensibili).\\
            I prodotti software utilizzati per eseguire queste attività vengono detti \textbf{Sniffer}.
        \end{minipage}
    \end{alertblock}
\end{frame}

\begin{frame}{SNIFFING DI RETE - ESEMPIO}
    \begin{columns}
        \column{.5\textwidth}
            \justifying
            \begin{enumerate}
                \item Iniziare la registrazione dei pacchetti tramite Wireshark
                \pause
                \item Collegarsi al sito: \href{http://testphp.vulnweb.com/login.php}{http://testphp.vulnweb.com}
                \pause
                \item Effettuare il login tramite protocollo non cifrato HTTP (non sicuro)
                \pause
                \item Ricercare nei pacchetti trasmessi un pacchetto HTTP POST (invio di dati)
                \pause
                \item Leggere l'username e la password in chiaro
            \end{enumerate}                        
        \column{.5\textwidth}
            \begin{figure}
                \href{https://www.wireshark.org/}{\includegraphics[width=\linewidth]{img/wireshark.png}}
                \caption{{creata con \href{https://chatgpt.com/}{ChatGPT}}}
            \end{figure}
    \end{columns}
\end{frame}

\begin{frame}{DENIAL OF SERVICE (DoS)}
    \begin{alertblock}{DEFINIZIONE}
        \begin{minipage}{0.98\linewidth}
            \justifying
            Attacco informatico in cui si fanno \textbf{esaurire le risorse di un sistema informatico} 
            che fornisce un servizio ai client, ad esempio un sito web su un server web, 
            fino a renderlo non più in grado di erogare il servizio ai client richiedenti.\\
            In un attacco Distributed Denial of Service (\textbf{DDoS}), il traffico dei dati in entrata 
            che inonda la vittima proviene da molte fonti diverse.\\
            \bigskip
            \tiny{\textbf{Video}}\\
            \tiny{\href{https://www.youtube.com/watch?v=ilhGh9CEIwM}{DDoS e Botnet}}
        \end{minipage}
    \end{alertblock}
\end{frame}

\begin{frame}{DENIAL OF SERVICE (DoS) - ESEMPIO}
    \begin{enumerate}
        \item Aprire il Task Manager e visualizzare l’attuale consumo di risorse;
        \pause
        \item Trovare l'indirizzo IP del Gateway predefinito: \textbf{ipconfig};
        \pause
        \item Eseguire un attacco DoS creando 20 terminali che inviano infiniti pacchetti di dati in parallelo 
        all'indirizzo IP del Gateway predefinito:\\ 
        \textbf{for /L \%i in (1,1,20) do start "" cmd /k "ping -t -l 65500 \textcolor{red}{ip\_gateway}"};
        \pause
        \item Aprire il Task Manager e visualizzare l’attuale consumo di risorse;
        \pause
        \item Terminare l'attacco: \textbf{taskkill /F /IM cmd.exe};
    \end{enumerate}                        
\end{frame}

\section{MALWARE}

\begin{frame}{RABBIT}
    \begin{alertblock}{DEFINIZIONE}
        \begin{minipage}{0.98\linewidth}
            \justifying
            Tipo di malware che \textbf{attacca le risorse del sistema} duplicando in continuazione la propria 
            immagine su disco, o attivando nuovi processi a partire dal proprio eseguibile, 
            in modo da \textbf{consumare tutte le risorse disponibili} sul sistema in pochissimo tempo. 
            Il nome si riferisce proprio alla prolificità di questo "infestante".\\
            \bigskip
            \tiny{\textbf{Esempio}}\\
            \tiny{\href{https://it.wikipedia.org/wiki/Fork_bomb}{Fork Bomb}}
            
        \end{minipage}
    \end{alertblock}
\end{frame}

\begin{frame}{RABBIT - ESEMPIO FORK BOMB}
    \begin{enumerate}
        \item Creare un file di testo RABBIT.txt;
        \pause
        \item Inserire la Fork Bomb nel file digitando: \textbf{start cmd /k echo BOMB \textbf{$\mid$} \%0}
        \pause
        \item Modificare l’estensione del file da .txt a .bat
        \pause
        \item Preparare un teminale con l'istruzione: \textbf{taskkill /F /IM cmd.exe} per stoppare il malware;
        \pause
        \item Eseguire il file RABBIT.bat e terminarlo con il comando preparato sul terminale.
    \end{enumerate}                        
\end{frame}

\begin{frame}{EFFETTUA UNA RICERCA SUI SEGUENTI MALWARE\\(definizione e un esempio per ogni tipologia)}
    \begin{itemize}
        \item \textbf{BACKDOOR};
        \item \textbf{RANSOMWARE};
        \item \textbf{TROJAN};
        \item \textbf{VIRUS};
        \item \textbf{WORM}.
    \end{itemize}                        
\end{frame}

\end{document}