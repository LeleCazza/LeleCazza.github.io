\documentclass[aspectratio=1610]{beamer}
%\documentclass[aspectratio=1610, handout]{beamer}
\usepackage[utf8]{inputenc}
\usepackage{ragged2e}
\usepackage{xcolor}
\usepackage[italian]{babel}
\usepackage{multirow}
\usetheme[progressbar=frametitle,titleformat=smallcaps]{metropolis}
\setbeamertemplate{frame numbering}[fraction]
\setbeamercovered{dynamic}
\definecolor{rosso}{RGB}{255, 0, 0}
\definecolor{giallo}{RGB}{254,212,23}
\hypersetup{colorlinks=true,linkcolor=black,urlcolor=rosso}
\setbeamercolor{palette primary}{fg=black, bg=giallo}
\setbeamercolor{background canvas}{bg=white}
\setbeamercolor{normal text}{fg=black}
\setbeamercolor{progress bar}{fg=rosso}
\setbeamercolor{framesubtitle}{fg=rosso}
\setbeamercolor{normal text .dimmed}{fg=giallo}
\setbeamercolor{block title alerted}{fg=rosso, bg=giallo}
\setbeamerfont{caption}{size=\tiny}
\setbeamerfont{caption name}{size=\tiny}
\setlength{\abovecaptionskip}{0pt}
\makeatletter
\metroset{block=fill}
\setlength{\metropolis@progressinheadfoot@linewidth}{1pt} 
\setlength{\metropolis@progressonsectionpage@linewidth}{1pt}
\setlength{\metropolis@titleseparator@linewidth}{1pt}
\makeatother

\title{CRITTOGRAFIA}
\subtitle{https, cifrari e applicazioni della crittografia}
\date{}
\institute{\textit{
        Fonti:
        \begin{itemize}
            \item[-] \href{https://www.treccani.it/enciclopedia/crittografia\_(Enciclopedia-della-Scienza-e-della-Tecnica)/}{Treccani}
            \item[-] \href{https://it.wikipedia.org/wiki/Crittografia_asimmetrica}{Wikipedia - crittografia asimmetrica}
            \item[-] \href{https://it.wikipedia.org/wiki/Non-fungible_token}{Wikipedia - NFT}
            \item[-] \href{https://www.fastweb.it/fastweb-plus/digital-dev-security/cose-la-blockchain-tutto-quello-che-devi-sapere/?fwp_login_type=3}{Fastweb Plus}
            \item[-] \href{https://www.ibm.com/it-it/think/topics/blockchain}{IBM - blockchain}
            \item[-] \href{https://www.kaspersky.it/resource-center/definitions/what-is-cryptocurrency}{Kaspersky - criptovalute} 
        \end{itemize}
    }
}

\begin{document}

\begin{frame}[plain, noframenumbering]
    \titlepage
\end{frame}

\section{CRITTOGRAFIA}

\begin{frame}{CRITTOGRAFIA}
    \begin{alertblock}{DEFINIZIONE}
        \begin{minipage}{0.98\linewidth}
            \justifying
            La crittografia è la disciplina che studia le \textbf{tecniche per trasformare un messaggio, 
            detto testo in chiaro, in un altro messaggio, detto testo cifrato, che risulta incomprensibile} 
            a chiunque non conosca tutti i dettagli della tecnica usata per la trasformazione. 
            Solo il legittimo destinatario del messaggio è in grado di effettuare l’operazione inversa 
            e di ottenere così dal testo cifrato il testo in chiaro originale. La trasformazione del testo 
            in chiaro in testo cifrato è detta \textbf{cifratura}, mentre la ricostruzione del testo in chiaro 
            a partire dal testo cifrato è detta \textbf{decifratura}. L’insieme delle operazioni che devono 
            essere effettuate durante la cifratura e la corrispondente decifratura prende il nome di codice 
            crittografico, o \textbf{cifrario}.\\
            \bigskip
            \tiny{\textbf{Curiosità}}\\
            \tiny{\href{https://it.wikipedia.org/wiki/Crittografia_end-to-end}{Crittografia end-to-end}}
        \end{minipage}
    \end{alertblock}
\end{frame}

\begin{frame}{HTTP vs HTTPS}
    \begin{columns}
        \column{.5\textwidth}
            \begin{figure}
                \href{http://www.flowgorithm.org/}{\includegraphics[width=\linewidth]{img/flowgorithm.png}}
                \caption{{creata con \href{https://chatgpt.com/}{ChatGPT}}}
            \end{figure}
        \column{.5\textwidth}
            \begin{figure}
                \href{https://www.netflix.com/it/}{\includegraphics[width=\linewidth]{img/netflix.png}}
                \caption{{creata con \href{https://www.canva.com/}{Canva}}}
            \end{figure}
    \end{columns}
\end{frame}

\section{CIFRARI SIMMETRICI}

\begin{frame}{CIFRARIO DI CESARE}
    \begin{alertblock}{DEFINIZIONE}
        \begin{minipage}{0.98\linewidth}
            \justifying
            Il cifrario di Cesare è uno dei più antichi algoritmi crittografici di cui si abbia traccia storica. 
            È un cifrario a \textbf{sostituzione monoalfabetica, in cui ogni lettera del testo in chiaro è sostituita, 
            nel testo cifrato, dalla lettera che si trova un certo numero di posizioni dopo nell'alfabeto}. 
            (nel caso del cifrario di Cesare, il numero di posizioni è 3). Questi tipi di cifrari sono detti anche 
            cifrari a sostituzione o cifrari a scorrimento a causa del loro modo di operare: la sostituzione avviene 
            lettera per lettera, scorrendo il testo dall'inizio alla fine.\\
            \bigskip
            \tiny{\textbf{Esempio}}\\
            \tiny{\href{http://www.crittologia.eu/critto/caesar.html}{Cifrario di Cesare}}
        \end{minipage}
    \end{alertblock}
\end{frame}

\begin{frame}{CIFRARIO DI VERNAM (OTP)}
    \begin{alertblock}{DEFINIZIONE}
        \begin{minipage}{0.98\linewidth}
            \justifying
            Esempio di cifrario a \textbf{chiave non riutilizzabile}, in inglese \textbf{One Time Pad} abbreviato in \textbf{OTP}. 
            Il cifrario di Vernam è perfetto, nel senso che il testo in chiaro e il testo cifrato sono del tutto indipendenti, 
            la conoscenza dell'uno non dà alcuna informazione sull'altro. È quindi del tutto al sicuro dagli attacchi della crittanalisi statistica. 
            \textbf{La chiave utilizzata per cifrare il messaggio deve essere lunga quanto il messaggio stesso} e non deve 
            essere mai riutilizzata, per questo viene chiamata One Time Password. Per ottenere il testo cifrato è sufficiente 
            eseguire un'operazione di \href{https://it.wikipedia.org/wiki/Disgiunzione\_esclusiva}{XOR} tra il testo in chiaro 
            e la chiave.\\
            \bigskip
            \tiny{\textbf{Esempio}}\\
            \tiny{\href{http://www.crittologia.eu/critto/vernam.phtml}{Cifrario OTP}}
        \end{minipage}
    \end{alertblock}
\end{frame}

\section{CIFRARIO ASIMMETRICO}

\begin{frame}{CIFRARIO ASIMMETRICO}
    \begin{alertblock}{DEFINIZIONE}
        \begin{minipage}{0.98\linewidth}
            \justifying
            La \textbf{crittografia asimmetrica} è un tipo di crittografia nel quale ad ogni attore 
            coinvolto nella comunicazione è associata una \textbf{coppia di chiavi}:
            \begin{itemize}
                \item La \textbf{chiave pubblica}, che deve essere distribuita;
                \item La \textbf{chiave privata}, personale e segreta.
            \end{itemize}
            La crittografia asimmetrica evita il problema classico della crittografia simmetrica connesso 
            alla necessità di uno scambio in modo sicuro dell'unica chiave utile alla cifratura/decifratura.
            Il meccanismo della crittografia asimmetrica si basa invece sulle seguenti assunzioni:
            \begin{itemize}
                \item La chiave privata non è ricavabile dalla chiave pubblica;
                \item Se con una delle due chiavi si cifra un messaggio, allora quest'ultimo sarà decifrato solo con l'altra.
            \end{itemize}
            \bigskip
            \tiny{\textbf{Esempio}}\\
            \tiny{\href{http://www.crittologia.eu/critto/rsa/rsa_demo.phtml}{Cifrario RSA}}
        \end{minipage}
    \end{alertblock}
\end{frame}

\begin{frame}{CRITTOGRAFIA: VIDEO}
    \begin{figure}
        \href{https://www.geopop.it/cose-la-crittografia-come-funziona-e-perche-serve-a-proteggere-la-confidenzialita-dei-dati/}{\includegraphics[width=\linewidth]{img/play.png}}
        \caption{{Fonte \href{https://www.geopop.it/cose-la-crittografia-come-funziona-e-perche-serve-a-proteggere-la-confidenzialita-dei-dati/}{Cos'è la crittografia, come funziona e perché serve a proteggere i dati e la confidenzialità (Geopop)}}}
    \end{figure}
\end{frame}

\section{APPLICAZIONI BASATE SULLA CRITTOGRAFIA}

\begin{frame}{BLOCKCHAIN}
    \begin{columns}
        \column{.5\textwidth}
            \begin{alertblock}{DEFINIZIONE}
                \begin{minipage}{0.96\linewidth}
                    \justifying
                    La \textbf{blockchain} funziona come un \textbf{database distribuito decentralizzato}, con dati 
                    archiviati su più computer, rendendolo resistente alle manomissioni. Le transazioni 
                    vengono convalidate attraverso un \textbf{meccanismo di consenso}, che garantisce l’accordo 
                    in tutta la rete. Nella tecnologia blockchain, \textbf{ogni transazione è raggruppata in blocchi}, 
                    che vengono poi collegati tra loro, formando una catena sicura e trasparente.\\
                    \bigskip
                    \tiny{\textbf{Video di approfondimento}}\\
                    \tiny{\href{https://telospiegovideo.it/blockchain-telospiego/}{Te lo spiego - Che cos'è è come funziona una blockchain}}
                \end{minipage}
            \end{alertblock}
        \column{.5\textwidth}
            \begin{figure}
                \includegraphics[width=\linewidth]{img/blockchain.png}
                \caption{{creata con \href{www.chatgpt.com}{ChatGPT}}}
                \bigskip
                \tiny{\textbf{Funzione di Hash}}\\
                \tiny{\href{https://it.wikipedia.org/wiki/Funzione_di_hash}{Wikipedia}}
            \end{figure}
    \end{columns}
\end{frame}

\begin{frame}{CRIPTOVALUTE}
    \begin{columns}
        \column{.5\textwidth}
            \begin{alertblock}{DEFINIZIONE}
                \begin{minipage}{0.96\linewidth}
                    \justifying
                    La \textbf{criptovaluta} è una forma di valuta digitale che \textbf{usa 
                    la crittografia per proteggere le transazioni}. Anziché avere un'autorità emittente o 
                    regolatrice centrale, le criptovalute \textbf{utilizzano una blockchain pubblica} per 
                    registrare le transazioni ed emettere nuove unità. 
                    Le unità di criptovaluta vengono \textbf{create tramite un processo chiamato mining}, 
                    che fa leva sull'elaborazione informatica per risolvere complicati problemi matematici 
                    da cui vengono generate le monete.\\
                    \bigskip
                    \tiny{\textbf{Video di approfondimento}}\\
                    \tiny{\href{https://telospiegovideo.it/che-cosa-sono-e-come-funzionano-i-bitcoin-telospiego/}{Te lo spiego - Che cosa sono e come funzionano i Bitcoin}}
                \end{minipage}
            \end{alertblock}
        \column{.5\textwidth}
            \begin{figure}
                \includegraphics[width=\linewidth]{img/criptovaluta.png}
                \caption{{creata con \href{www.chatgpt.com}{ChatGPT}}}
                \bigskip
                \tiny{\textbf{Prezzi delle criptovalute}}\\
                \tiny{\href{https://www.coingecko.com/}{Andamento delle principali criptovalute}}\\
                \tiny{\href{https://www.coingecko.com/it/categories/meme-token}{Andamento dei principali Meme Coin}}
            \end{figure}
    \end{columns}
\end{frame}

\begin{frame}{NON-FUNGIBLE TOKEN (NFT)}
    \begin{columns}
        \column{.5\textwidth}
            \begin{alertblock}{DEFINIZIONE}
                \begin{minipage}{0.96\linewidth}
                    \justifying
                    Un non-fungible token (\textbf{NFT}) è un tipo speciale di token, 
                    che rappresenta un \textbf{bene unico non fungibile}. 
                    A differenza di un bene fungibile (come una banconota), non può essere scambiato 
                    uno a uno con altri beni dello stesso tipo in modo indistinto. Gli NFT sono un tipo 
                    di bene non fungibile che possiede dati unici e possono essere \textbf{utilizzati per 
                    registrare e verificare la proprietà tramite la tecnologia della blockchain}.\\
                    \bigskip
                    \tiny{\textbf{Esempio di NFT}}\\
                    \tiny{\href{https://www.cryptokitties.co/}{CryptoKitties}}
                \end{minipage}
            \end{alertblock}
        \column{.5\textwidth}
            \begin{figure}
                \includegraphics[width=\linewidth]{img/nft.png}
                \caption{{fonte \href{https://it.wikipedia.org/wiki/Non-fungible_token}{Wikipedia}}}
            \end{figure}
    \end{columns}
\end{frame}

\begin{frame}{CYPHERPUNKS}
    \begin{minipage}{0.98\linewidth}
        \centering
        \Large
        \textit{``Privacy is necessary for an open society in the electronic age. 
        Privacy is not secrecy. A private matter is something one doesn't want the whole 
        world to know, but a secret matter is something one doesn't want anybody to know. 
        Privacy is the power to selectively reveal oneself to the world.''}\\
    \end{minipage}\\
    \bigskip
    \tiny{\textbf{Curiosità}}\\
    \tiny{\href{https://it.wikipedia.org/wiki/Cypherpunk}{Chi sono i Cypherpunks?}}\\
    \tiny{\href{https://www.activism.net/cypherpunk/manifesto.html}{A Cypherpunk's Manifesto}}
\end{frame}
\end{document}