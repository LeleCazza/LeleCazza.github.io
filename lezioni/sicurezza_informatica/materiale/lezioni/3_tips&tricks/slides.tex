% FONTE TEMA https://github.com/matze/mtheme
\documentclass[aspectratio=1610]{beamer}
%\documentclass[aspectratio=1610, handout]{beamer}
\usepackage[utf8]{inputenc}
\usepackage{ragged2e}
\usepackage{xcolor}
\usepackage[italian]{babel}
\usepackage{multirow}
\usepackage{silence}
\WarningFilter{beamer}{}
\WarningFilter{metropolis}{}
\usetheme[progressbar=frametitle,titleformat=smallcaps]{metropolis}
\setbeamertemplate{frame numbering}[fraction]
\setbeamercovered{dynamic}
\definecolor{rosso}{RGB}{255, 0, 0}
\definecolor{giallo}{RGB}{254,212,23}
\hypersetup{colorlinks=true,linkcolor=black,urlcolor=rosso}
\setbeamercolor{palette primary}{fg=black, bg=giallo}
\setbeamercolor{background canvas}{bg=white}
\setbeamercolor{normal text}{fg=black}
\setbeamercolor{progress bar}{fg=rosso}
\setbeamercolor{framesubtitle}{fg=rosso}
\setbeamercolor{normal text .dimmed}{fg=giallo}
\setbeamercolor{block title alerted}{fg=rosso, bg=giallo}
\setbeamerfont{caption}{size=\tiny}
\setbeamerfont{caption name}{size=\tiny}
\setlength{\abovecaptionskip}{0pt}
\makeatletter
\metroset{block=fill}
\setlength{\metropolis@progressinheadfoot@linewidth}{1pt} 
\setlength{\metropolis@progressonsectionpage@linewidth}{1pt}
\setlength{\metropolis@titleseparator@linewidth}{1pt}
\makeatother

\title{TIPS \& TRICKS}
\subtitle{Buone pratiche per migliorare la sicurezza informatica}
\date{}
\institute{\textit{
        Fonti:
        \begin{itemize}
            \item[-] \href{https://support.google.com/accounts/answer/32040?hl=it}{Google - creare una password efficace}
            \item[-] \href{https://it.wikipedia.org/wiki/Password}{Wikipedia - Password}
            \item[-] \href{https://www.geopop.it/cosa-sono-i-password-manager-come-funzionano-e-quanto-sono-sicuri-i-gestori-di-password/}{Geopop - Password Manager} 
            \item[-] \href{https://www.fastweb.it/fastweb-plus/digital-dev-security/passkey-cose-e-come-funziona/?fwp_login_type=3}{Fastweb Plus - Passkey}
        \end{itemize}
    }
}

\begin{document}

\begin{frame}[plain, noframenumbering]
    \titlepage
\end{frame}

\section{PASSWORD}

\begin{frame}{PASSWORD}
    \begin{alertblock}{DEFINIZIONE}
        \begin{minipage}{0.98\linewidth}
            \justifying
            Una \textbf{password} è una sequenza di caratteri alfanumerici e di simboli utilizzata 
            per \textbf{accedere in modo esclusivo a una risorsa informatica}. Si parla più 
            propriamente di \textbf{passphrase} se la chiave è costituita da una frase o da 
            una sequenza sufficientemente lunga di caratteri (non meno di 20/30).\\
            Una password efficace può essere facile da ricordare per te, ma quasi impossibile da 
            indovinare per gli altri.\\
            \bigskip
            \tiny{\textbf{Curiosità}}\\
            \tiny{\href{https://it.wikipedia.org/wiki/Automated_Password_Generator}{Automated Password Generator (APG)}}
        \end{minipage}
    \end{alertblock}
\end{frame}

\begin{frame}{SUGGERIMENTI GOOGLE PER CREARE UNA PASSWORD EFFICACE}
    \begin{itemize}
        \item \textbf{Crea una password univoca}: Usa una password diversa per ogni account importante;
        \pause
        \item \textbf{Crea una password più lunga e facile da ricordare}: imposta una password di almeno 12 caratteri. 
        Prova a usare:
        \begin{itemize}
            \item Testo di una canzone o una poesia
            \item Una citazione significativa di un film o un discorso
            \item Un passaggio di un libro
            \item Una serie di parole significative per te
            \item Un'abbreviazione: crea una password usando la prima lettera di ogni parola di una frase
        \end{itemize}
        Evita di scegliere password che possano essere intuite da persone che conosci o che guardano informazioni 
        facilmente accessibili (come il tuo profilo sui social media);
    \end{itemize}
\end{frame}

\begin{frame}{SUGGERIMENTI GOOGLE PER CREARE UNA PASSWORD EFFICACE}
    \begin{itemize}
        \item \textbf{Evita informazioni personali}: Evita di creare password usando informazioni note 
        ad altre persone o facilmente indovinabili. Esempi:
        \begin{itemize}
            \item Il tuo nickname o le tue iniziali;
            \item Il nome di tuo figlio o del tuo animale domestico;
            \item Compleanni o anni importanti;
            \item Il nome della tua via;
            \item Numeri del tuo indirizzo;
            \item Il tuo numero di telefono.
        \end{itemize}
        \pause
        \item \textbf{Evita parole o sequenze comuni}: Evita parole, espressioni e sequenze facili da indovinare. Esempi:
        \begin{itemize}
            \item Parole e frasi ovvie come "password" e "fammiaccedere";
            \item Sequenze come "abcd" o "1234";
            \item Sequenze della tastiera come "qwerty" o "qazwsx".
        \end{itemize}
        \pause
        \item \textbf{Nascondi le password scritte}: Se devi annotarti la password, non lasciarla sul 
        computer o sulla scrivania. Assicurati di conservare le password scritte in posti 
        segreti o chiusi a chiave.
    \end{itemize}
\end{frame}

\begin{frame}{PASSWORD MANAGER}
    \begin{columns}
        \column{.55\textwidth}
            \begin{alertblock}{DEFINIZIONE}
                \begin{minipage}{0.96\linewidth}
                    \justifying
                    Applicazione progettata per \textbf{gestire e automatizzare tutte le credenziali di accesso ai propri 
                    account per diversi siti e servizi}. L’obiettivo principale è facilitare la creazione e l'utilizzo 
                    di password complesse e uniche per ogni account, senza doverle ricordare tutte.
                    \begin{itemize}
                        \item Password manager locali \\(esempio: \href{https://keepassxc.org/}{KeePassXC})
                        \item Password manager basati sul cloud: \\(esempio: \href{https://bitwarden.com/}{Bitwarden})
                        \item Password manager dei browser: \\(esempio: \href{https://passwords.google.com/}{Google Chrome})
                        \item Password manager basati su hardware
                    \end{itemize}
                \end{minipage}
            \end{alertblock}
        \column{.45\textwidth}
            \begin{figure}
                \includegraphics[width=\linewidth]{img/passwordmanager.png}
                \caption{{creata con \href{gemini.google.com}{Gemini}}}
            \end{figure}
    \end{columns}
\end{frame}

\begin{frame}{PASSKEY}
    \begin{alertblock}{DEFINIZIONE}
        \begin{minipage}{0.98\linewidth}
            \justifying
            Una \textbf{passkey} è un metodo di autenticazione alternativo alle password, che permette di 
            accedere ad app, programmi, siti web o servizi digitali. L’autenticazione passkey \textbf{ricorre 
            ai dati biometrici dell’utente}. Quando si accede, \textbf{il dispositivo utilizza i dati 
            biometrici o un PIN per verificare che si tratti dell’utente legittimo}. 
            Dopo la verifica, il dispositivo invia una risposta sicura, generata a partire da 
            tale passkey univoca, al servizio richiesto.\\
            \bigskip
            \tiny{\textbf{Approfondimento}}\\
            \tiny{\href{https://www.kaspersky.it/blog/full-guide-to-passkeys-in-2025-part-1/29790/}{Guida alle passkey}}
        \end{minipage}
    \end{alertblock}
\end{frame}

\begin{frame}{AUTENTICAZIONE A DUE FATTORI (2FA)}
    \centering
    \begin{tabular}{c||c}
        \textbf{TIPOLOGIA} & \textbf{DESCRIZIONE} \\
        \hline
        \hline
        \pause
        \multirow{2}{5.5cm}{\textbf{MESSAGGIO TELEFONO}} & \multirow{2}{8cm}{OTP inviato per SMS al numero di telefono specificato dall’utente in fase di registrazione} \\
        & \\
        \hline
        \pause
        \multirow{2}{5.5cm}{\textbf{INDIRIZZO MAIL}} & \multirow{2}{8cm}{OTP inviato per MAIL all’indirizzo specificato dall’utente in fase di registrazione} \\
        & \\
        \hline
        \pause
        \multirow{2}{5.5cm}{\textbf{DISPOSITIVO HARDWARE}} & \multirow{2}{8cm}{OTP generato tramite hardware specifico} \\
        & \\
        \hline
        \pause
        \multirow{2}{5.5cm}{\textbf{APPLICAZIONE}} & \multirow{2}{8cm}{OTP inviato tramite app installata sullo smartphone (esempio: \href{https://getaegis.app/}{Aegis Authenticator})} \\
        & \\
        \hline
    \end{tabular}
\end{frame}

\begin{frame}{HAVE I BEEN PWNED?}
    \begin{figure}
        \href{https://haveibeenpwned.com/}{\includegraphics[width=.5\linewidth]{img/haveibeenpwned.png}}
        \caption{{creata con \href{gemini.google.com}{Gemini}}}
    \end{figure}
\end{frame}

\section{GESTIONE ACCOUNT}

\begin{frame}{GESTIONE ACCOUNT}
    \begin{alertblock}{BUONE NORME DA SEGUIRE}
        \begin{minipage}{0.98\linewidth}
            \justifying
            COMPITO: Esegui una revisione completa di tutti gli account online e applicazioni che possiedi e 
            delle password associate ad essi, installando un password manager.
            \bigskip
            \begin{enumerate}
                \pause
                \item Eliminare tutti gli account inutilizzati;
                \pause
                \item Eliminare tutte le applicazioni inutilizzate;
                \pause
                \item Cambiare le password a tutti gli account rimasti e utilizzare un password manager;
                \pause
                \item Disattivare dalle impostazioni di ogni account/applicazione rimasta tutti i 
                permessi/opzioni di condivisione dati non strettamente necessari.
            \end{enumerate}
        \end{minipage}
    \end{alertblock}
\end{frame}

\begin{frame}{GESTIONE ACCOUNT}
    \begin{alertblock}{BUONE NORME DA SEGUIRE}
        \begin{minipage}{0.98\linewidth}
            \begin{enumerate}
                \item Prima di iscriverti a un nuovo servizio online, verifica se ne hai davvero bisogno 
                utilizzando una mail temporanea per la registrazione (esempio: \href{https://10minutemail.com/}{10 Minute Mail});
                \pause
                \item Quando ti iscrivi ad un sito web o ad un servizio online, rimuovi tutte le spunte 
                relative a newsletter, pubblicità o condivisione dei tuoi dati con terze parti, inserendo 
                solamente i consensi o i dati obbligatori (*);
                \pause
                \item Pulisci la posta elettronica disiscrivendoti dalle newsletter/spam non desiderati 
                (in genere cliccando sul link "Unsubscribe" in fondo alla mail).
            \end{enumerate}
        \end{minipage}
    \end{alertblock}
\end{frame}

\section{PRIVACY}

\begin{frame}{PRIVACY}
    \begin{alertblock}{TIPS\&TRICKS}
        \begin{minipage}{0.98\linewidth}
            \begin{enumerate}
                \item Utilizza software alternativo ai classici per ottenere una navigazione più privata e sicura 
                (esempio: \href{https://www.privacytools.io}{PrivacyTools});
                \pause
                \item Utilizza un intermediario Proxy o una VPN per nascondere il tuo indirizzo IP reale 
                e visualizzare pagine web in modo ``più anonimo'' (esempio: \href{https://hide.me/it/proxy}{Hide me});
                \pause
                \item Utilizza un account ospite con navigazione in incognito per non memorizzare sul dispositivo 
                cookies, account e cronologia delle ricerche effettuate;
                \pause
                \item \textbf{Non dare mai per scontato che sia impossibile violare i dispositivi \href{https://it.wikipedia.org/wiki/Internet_delle_cose}{IoT}.}
            \end{enumerate}
            \bigskip
            \tiny{\textbf{Approfondimento}}\\
            \tiny{\href{https://www.redhat.com/it/topics/security/security-for-iot-devices}{La sicurezza per i dispositivi IoT}}
        \end{minipage}
    \end{alertblock}
\end{frame}

\section{INFORMAZIONE}

\begin{frame}{INFORMAZIONE}
    \begin{alertblock}{FONTI AFFIDABILI}
        \begin{minipage}{0.98\linewidth}
            \justifying
            Esistono numerose fonti online affidabili per rimanere aggiornati sulle ultime novità 
            e tendenze nel campo della sicurezza informatica. Di seguito alcune delle più rinomate e rispettate:
            \bigskip
            \begin{itemize}
                \pause
                \item \href{https://www.acn.gov.it/portale/home}{\textbf{ACN}}: Agenzia per la Cybersicurezza Nazionale italiana;
                \pause
                \item \href{https://www.ncsc.admin.ch/ncsc/it/home.html}{\textbf{NCSC}}: National Cyber Security Centre Svizzera;
                \pause
                \item \href{https://www.cybersecitalia.it/}{\textbf{CyberSecurity Italia}}: Quotidiano online dedicato alla sicurezza informatica;
                \pause
                \item \href{https://www.wired.it/topic/cybersecurity/}{\textbf{Wired}}: Sezione Cybersecurity della rivista di tecnologia Wired.
            \end{itemize}
        \end{minipage}
    \end{alertblock}
\end{frame}

\end{document}