\documentclass[aspectratio=1610]{beamer}
%\documentclass[aspectratio=1610, handout]{beamer}
\usepackage[utf8]{inputenc}
\usepackage{ragged2e}
\usepackage{xcolor}
\usepackage[italian]{babel}
\usepackage{multirow}
\usetheme[progressbar=frametitle,titleformat=smallcaps]{metropolis}
\setbeamertemplate{frame numbering}[fraction]
\setbeamercovered{dynamic}
\definecolor{rosso}{RGB}{255, 0, 0}
\definecolor{giallo}{RGB}{254,212,23}
\hypersetup{colorlinks=true,linkcolor=black,urlcolor=rosso}
\setbeamercolor{palette primary}{fg=black, bg=giallo}
\setbeamercolor{background canvas}{bg=white}
\setbeamercolor{normal text}{fg=black}
\setbeamercolor{progress bar}{fg=rosso}
\setbeamercolor{framesubtitle}{fg=rosso}
\setbeamercolor{normal text .dimmed}{fg=giallo}
\setbeamercolor{block title alerted}{fg=rosso, bg=giallo}
\setbeamerfont{caption}{size=\tiny}
\setbeamerfont{caption name}{size=\tiny}
\setlength{\abovecaptionskip}{0pt}
\makeatletter
\metroset{block=fill}
\setlength{\metropolis@progressinheadfoot@linewidth}{1pt} 
\setlength{\metropolis@progressonsectionpage@linewidth}{1pt}
\setlength{\metropolis@titleseparator@linewidth}{1pt}
\makeatother

\title{LISTA DELLE FUNZIONI}
\subtitle{Fogli di calcolo}
\date{}
\institute{}

\begin{document}

\begin{frame}[plain, noframenumbering]
    \titlepage
\end{frame}

\begin{frame}{FORMATI BASE}
    \begin{itemize}
        \justifying
        \item \textbf{VALUTA (€)} esempio: 23,01€ (ventitré euro e un centesimo)
        \item \textbf{DURATA (hh.mm.ss)} esempio: 2.30.00 (due ore, 30 minuti, 0 secondi)
        \item \textbf{PERCENTUALE (\%)} esempio: 15\% (quindici percento)
    \end{itemize}
\end{frame}

\begin{frame}{OPERATORI BASE}
    \begin{itemize}
        \justifying
        \item \textbf{Somma tra due celle:} = cella1 + cella2
        \item \textbf{Sottrazione tra due celle:} = cella1 - cella2
        \item \textbf{Moltiplicazione tra due celle:} = cella1 * cella2
        \item \textbf{Divisione tra due celle:} = cella1 / cella2
    \end{itemize}
\end{frame}

\begin{frame}{FUNZIONI BASE}
    \begin{itemize}
        \justifying
        \item \textbf{Somma tra N celle:} = SOMMA (cella1 : cellaN)
        \item \textbf{Media tra N celle:} = MEDIA (cella1 : cellaN)
        \item \textbf{Massimo tra N celle:} = MAX (cella1 : cellaN)
        \item \textbf{Minimo tra N celle:} = MIN (cella1 : cellaN)
    \end{itemize}
\end{frame}

\begin{frame}{RIFERIMENTI}
    \begin{itemize}
        \justifying
        \item \textbf{Relativi:} riferimenti semplici alle celle (es. A1, B2, C3). I 
        riferimenti relativi cambiano in base alla direzione del completamento automatico.
    \item \textbf{Assoluti:} riferimenti assoluti alle celle (es. \$A\$1, \$B\$2, \$C\$3). 
        I riferimenti assoluti in caso di completamenti automatici rimangono invariati.
    \end{itemize}
\end{frame}

\begin{frame}{OPERATORI DI CONFRONTO}
    \begin{itemize}
        \justifying
        \item \textbf{Confronta se la prima cella è uguale alla seconda cella:} \\ = cella1 = cella2
        \item \textbf{Confronta se la prima cella è maggiore della seconda cella:} \\ = cella1 \textgreater{} cella2
        \item \textbf{Confronta se la prima cella è maggiore o uguale alla seconda cella:} \\ = cella1 \textgreater= cella2
        \item \textbf{Confronta se la prima cella è minore della seconda cella:} \\ = cella1 \textless{} cella2
        \item \textbf{Confronta se la prima cella è minore o uguale alla seconda cella:} \\ = cella1 \textless= cella2
    \end{itemize}
\end{frame}

\begin{frame}{FUNZIONI LOGICHE}
    \begin{itemize}
        \justifying
        \item \textbf{Restituisce VERO se tutte le condizioni hanno valore VERO, 
        FALSO altrimenti:} = E (condizione1 ; condizione2 ; ecc.)
        \item \textbf{Restituisce VERO se una delle condizioni ha valore VERO, FALSO 
        altrimenti:} = O (condizione1 ; condizione2 ; ecc.)
        \item \textbf{Inverte in valore logico dell\textquotesingle argomento, 
        restituisce FALSO per un argomento VERO e VERO per un argomento FALSO:} = NON (argomento)
    \end{itemize}
\end{frame}

\begin{frame}{FUNZIONE CASUALE E DI SCELTA}
    \begin{itemize}
        \justifying
        \item \textbf{Numero casuale compreso tra due valori:} = CASUALE.TRA (num1 ; num2)
        \item \textbf{Seleziona un valore o un\textquotesingle azione da eseguire da 
        un elenco di valori in base a un\textquotesingle indice:} = SCEGLI (indice ; valore1 ; valore2 ; ecc.)
    \end{itemize}
\end{frame}

\begin{frame}{FORMATTAZIONE CONDIZIONALE}
    \begin{itemize}
        \justifying
        \item \textbf{Formula preimpostata}: formatta le celle desiderate in base a 
        funzioni preimpostate. La creazione delle regole di formattazione deve 
        essere impostata scegliendo la funzione preimpostata da utilizzare 
        tramite gli strumenti messi a disposizione dal software per la 
        gestione del foglio di calcolo.
        \item \textbf{Formula personalizzata}: formatta le celle desiderate in base 
        a funzioni personalizzate. La creazione delle regole di formattazione 
        deve essere impostata personalizzando la funzione da utilizzare 
        tramite gli strumenti messi a disposizione dal software per la 
        gestione del foglio di calcolo.
    \end{itemize}
\end{frame}

\begin{frame}{FUNZIONI CONDIZIONALI}
    \begin{itemize}
        \justifying
        \item \textbf{Nel caso la condizione sia verificata, effettua i comandi 
        presenti nella sezione vero, in alternativa, effettua i comandi 
        presenti nella sezione falso:} \\
        = SE ( condizione ; vero ; falso )
        \item \textbf{Somma le celle specificate secondo una condizione assegnata a un determinato intervallo di celle:} 
        = SOMMA.SE ( intervallo\_da\_considerare ; criterio\_per\_la\_somma ; intervallo\_effettivo\_da\_sommare)
        \item \textbf{Conta il numero di celle di un determinato intervallo che soddisfano la condizione espressa nel campo criterio:} 
        = CONTA.SE ( intervallo ; criterio )
    \end{itemize}
\end{frame}

\begin{frame}{FUNZIONI AGGREGATE CONDIZIONALI}
    \begin{itemize}
        \justifying
        \item \textbf{Somma le celle presenti in un intervallo specificato solamente 
        nel caso in cui vengano soddisfatte un determinato insieme di 
        condizioni su determinati intervalli di celle:}  = SOMMA.PIÙ.SE ( intervallo\_effettivo\_da\_sommare ; 
        intervallo\_criterio\_1 ; criterio\_1 ; intervallo\_criterio\_2 ; criterio\_2 ; ecc. )
        \item \textbf{Conta il numero di celle presenti in intervalli specificati 
        solamente nel caso in cui vengano soddisfatte un determinato insieme 
        di condizioni su determinati intervalli di celle:} =CONTA.PIÙ.SE ( intervallo\_criterio\_1 ; 
        criterio\_1 ; intervallo\_criterio\_2 ; criterio\_2 ; ecc.)
    \end{itemize}
\end{frame}
\end{document}