\documentclass[aspectratio=1610]{beamer}
\usepackage[utf8]{inputenc}
\usepackage{ragged2e}
\usepackage{xcolor}
\usepackage[italian]{babel}
\usetheme[progressbar=frametitle,titleformat=smallcaps]{metropolis}
\setbeamertemplate{frame numbering}[fraction]
\setbeamercovered{dynamic}
\definecolor{rosso}{RGB}{255, 0, 0}
\definecolor{giallo}{RGB}{254,212,23}
\hypersetup{colorlinks=true,linkcolor=black,urlcolor=rosso}
\setbeamercolor{palette primary}{fg=black, bg=giallo}
\setbeamercolor{background canvas}{bg=white}
\setbeamercolor{normal text}{fg=black}
\setbeamercolor{progress bar}{fg=rosso}
\setbeamercolor{framesubtitle}{fg=rosso}
\setbeamercolor{normal text .dimmed}{fg=giallo}
\setbeamercolor{block title alerted}{fg=rosso, bg=giallo}
\setbeamerfont{caption}{size=\tiny}
\setbeamerfont{caption name}{size=\tiny}
\setlength{\abovecaptionskip}{0pt}
\makeatletter
\metroset{block=fill}
\setlength{\metropolis@progressinheadfoot@linewidth}{1pt} 
\setlength{\metropolis@progressonsectionpage@linewidth}{1pt}
\setlength{\metropolis@titleseparator@linewidth}{1pt}
\makeatother

\title{TRASMISSIONE DATI}
\subtitle{wired VS wireless, tipologie di trasmissione}
\date{}
\institute{\textit{
        Fonti:
        \begin{itemize}
            \item[-] \href{https://it.wikipedia.org/wiki/FTTx}{Wikipedia}
            \item[-] \href{https://www.edises.it/concorsi/scienze-e-tecnologie-informatiche-2020.html}{Manuale scienze e tecnologie informatiche}
        \end{itemize}
    }
}

\begin{document}

\begin{frame}[plain, noframenumbering]
    \titlepage
\end{frame}


\section{WIRED VS WIRELESS}

\begin{frame}{VELOCIT\'A DI TRASMISSIONE}
    \begin{alertblock}{DEFINIZIONE}
        \begin{minipage}{0.98\linewidth}
            \justifying
            In informatica e telecomunicazioni, \textbf{la velocità di trasmissione} 
            (detta anche \textbf{banda} o \textbf{bit-rate}), 
            è la grandezza indicante la quantità di informazione trasferita 
            attraverso un canale di comunicazione in un dato intervallo di tempo. 
            L'unità di misura associata è il bit per secondo (\textbf{bps}).\\
            \bigskip
            \tiny{\textbf{Fonte}}\\
            \tiny{\href{https://it.wikipedia.org/wiki/Velocit\%C3\%A0\_di\_trasmissione}{Wikipedia}}
        \end{minipage}
    \end{alertblock}
\end{frame}

\begin{frame}{MEZZI DI TRASMISSIONE DATI (\href{https://www.fastweb.it/fastweb-plus/digital-magazine/velocita-connessione-internet-c-e-un-nuovo-record/}{record speed}, \href{https://misurainternet.it/misura-speedtest/}{speed test})}
    \begin{center}
        \centering
        \setlength{\tabcolsep}{30pt}
        \begin{tabular}{c|c}
            \textbf{TRASMISSIONE WIRED} & \textbf{BIT-RATE} \\
            \hline
            \hline
            \uncover<1->{Doppino telefonico (ADSL)} & \uncover<1->{fino a 24 Mbps (effettivo)} \\
            \hline
            \uncover<2->{Cavo UTP/STP (Ethernet)} & \uncover<2->{4 Mbps - 40Gbps (teorico)} \\ 
            \hline
            \uncover<3->{Fibra Ottica FTTC (Cabinet)} & \uncover<3->{50 Mbps - 200 Mbps (effettivo)} \\
            \hline
            \uncover<4->{Fibra Ottica FTTH (Home)} & \uncover<4->{50 Mbps - 2,5Gbps (effettivo)} \\
            \hline
        \end{tabular}
    \end{center}
\end{frame}

\begin{frame}{FREQUENZA}
    \begin{alertblock}{DEFINIZIONE}
        \begin{minipage}{0.98\linewidth}
            \justifying
            In fisica \textbf{la frequenza} di un fenomeno che presenta un andamento costituito da 
            eventi che nel tempo si ripetono (identici o quasi identici), viene data dal numero 
            degli eventi che vengono ripetuti in una data unità di tempo. Nel misurare la frequenza 
            di onde elettromagnetiche (come le onde radio o la luce), la frequenza in \textbf{hertz} è il numero 
            di cicli della forma d'onda ripetitiva per secondo.\\
            \bigskip
            \tiny{\textbf{Fonte}}\\
            \tiny{\href{https://it.wikipedia.org/wiki/Frequenza}{Wikipedia}}
        \end{minipage}
    \end{alertblock}
\end{frame}

\begin{frame}{SPETTRO ELETTROMAGNETICO}
    \begin{columns}
        \column{.65\textwidth}
        \begin{figure}
            \includegraphics[width=\textwidth]{img/Spettro_Elettromagnetico.png}
            \caption{{creata con \href{https://www.canva.com/}{Canva}}}
        \end{figure}
    \end{columns}
\end{frame}

\section{SEZIONE 1: IMMAGINE CENTRATA}

\begin{frame}{TITOLO SLIDE}
%    \begin{columns}
%        \column{.5\textwidth}
%        \begin{figure}
%            \includegraphics[width=\linewidth]{img/immagine.png}
%            \caption{{creata con \href{LINK SOFTWARE DI CREAZIONE}{NOME SOFTWARE}}}
%        \end{figure}
%    \end{columns}
\end{frame}

\section{SEZIONE 2: ELENCO PUNTATO}

\begin{frame}{TITOLO SLIDE}
%    \begin{itemize}
%        \item \textbf{NOME}: descrizione;
%        \pause
%        \item \textbf{NOME}: descrizione;
%        \pause
%        \item \textbf{NOME}: descrizione;
%    \end{itemize}
\end{frame}

\section{SEZIONE 3: TABELLA}

\begin{frame}{TITOLO SLIDE}
%    \centering
%    \begin{tabular}{c||c}
%        \textbf{TITOLO COLONNA 1} & \textbf{TITOLO COLONNA 2} \\
%        \hline
%        \hline
%        \pause
%        ELEMENTO SINISTRA & ELEMENTO DESTRA \\
%        \hline
%        \pause
%        ELEMENTO SINISTRA & ELEMENTO DESTRA \\
%        \hline
%    \end{tabular}
\end{frame}

\section{SEZIONE 4: SEQUENZA IMMAGINI}

\begin{frame}{TITOLO SLIDE}
%   \only<1>{\begin{figure}
%       \includegraphics[width=\linewidth]{img/immagine1.png}
%       \caption{{creata con \href{LINK SOFTWARE DI CREAZIONE}{NOME SOFTWARE}}}
%   \end{figure}}
%   \only<2>{\begin{figure}
%       \includegraphics[width=\linewidth]{img/immagine2.png}
%       \caption{{creata con \href{LINK SOFTWARE DI CREAZIONE}{NOME SOFTWARE}}}
%   \end{figure}}
%   \only<3>{\begin{figure}
%       \includegraphics[width=\linewidth]{img/immagine3.png}
%       \caption{{creata con \href{LINK SOFTWARE DI CREAZIONE}{NOME SOFTWARE}}}
%   \end{figure}}
\end{frame}

\section{SEZIONE 5: DEFINIZIONE}

\begin{frame}{TITOLO SLIDE}
    \begin{alertblock}{DEFINIZIONE}
        \begin{minipage}{0.98\linewidth}
            \justifying
            % testo definizione\\
            %\bigskip
            %\tiny{\textbf{Curiosità}}\\
            %\tiny{\href{Link curiosità}{Titolo curiosità}}
        \end{minipage}
    \end{alertblock}
\end{frame}

\section{SEZIONE 6: DUE COLONNE (testo, immagine)}

\begin{frame}{TITOLO SLIDE}
    \begin{columns}
        \column{.5\textwidth}
            \justifying
%           testo a sinistra dell'immagine
%           \bigskip
%           \tiny{\textbf{curiosità}}\\
%           \tiny{LINK CURIOSITA'}{TITOLO CURIOSITA'}}
        \column{.5\textwidth}
%           \begin{figure}
%               \includegraphics[width=\linewidth]{img/immagine1.png}
%               \caption{{creata con \href{LINK SOFTWARE DI CREAZIONE}{NOME SOFTWARE}}}
%           \end{figure}
    \end{columns}
\end{frame}

\end{document}