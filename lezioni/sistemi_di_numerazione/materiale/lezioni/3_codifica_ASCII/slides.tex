% FONTE TEMA https://github.com/matze/mtheme
\documentclass[aspectratio=1610]{beamer}
%\documentclass[aspectratio=1610, handout]{beamer}
\usepackage[utf8]{inputenc}
\usepackage{ragged2e}
\usepackage{xcolor}
\usepackage[italian]{babel}
\usepackage{multirow}
\usepackage{silence}
\WarningFilter{beamer}{}
\WarningFilter{metropolis}{}
\usetheme[progressbar=frametitle,titleformat=smallcaps]{metropolis}
\setbeamertemplate{frame numbering}[fraction]
\setbeamercovered{dynamic}
\definecolor{rosso}{RGB}{255, 0, 0}
\definecolor{giallo}{RGB}{254,212,23}
\hypersetup{colorlinks=true,linkcolor=black,urlcolor=rosso}
\setbeamercolor{palette primary}{fg=black, bg=giallo}
\setbeamercolor{background canvas}{bg=white}
\setbeamercolor{normal text}{fg=black}
\setbeamercolor{progress bar}{fg=rosso}
\setbeamercolor{framesubtitle}{fg=rosso}
\setbeamercolor{normal text .dimmed}{fg=giallo}
\setbeamercolor{block title alerted}{fg=rosso, bg=giallo}
\setbeamerfont{caption}{size=\tiny}
\setbeamerfont{caption name}{size=\tiny}
\setlength{\abovecaptionskip}{0pt}
\makeatletter
\metroset{block=fill}
\setlength{\metropolis@progressinheadfoot@linewidth}{1pt} 
\setlength{\metropolis@progressonsectionpage@linewidth}{1pt}
\setlength{\metropolis@titleseparator@linewidth}{1pt}
\makeatother

\title{RAPPRESENTAZIONE CARATTERI}
\subtitle{La codifica ASCII}
\date{}
\institute{\textit{
        Fonti:
        \begin{itemize}
            \item[-] \href{https://catalogo.sanoma.it/si-op-104157-dal-bit-all-intelligenza-artificiale.html}{Dal BIT all'INTELLIGENZA ARTIFICIALE}
        \end{itemize}
    }
}

\begin{document}

\begin{frame}[plain, noframenumbering]
    \titlepage
\end{frame}

\begin{frame}{TITOLO SLIDE}
    \begin{alertblock}{DEFINIZIONE}
        \begin{minipage}{0.98\linewidth}
            \justifying
            Per rappresentare i caratteri (lettere, numeri, simboli), deve essere utilizzato un 
            \textbf{sistema di codifica} che associ ad ogni carattere un numero. 
            L'\textbf{ASCII} (American Standard Code for Information Interchange) è uno dei più 
            diffusi sistemi di codifica, che associa ad ogni carattere un numero compreso tra 0 e 127 (7 bit).\\
            \pause
            Siccome l'ASCII non è in grado di codificare tutti i caratteri di tutte le lingue mondiali, 
            sono stati sviluppati sistemi di codifica più estesi, i più utilizzati sono:
            \begin{itemize}
                \pause
                \item \href{https://it.wikipedia.org/wiki/ASCII_esteso}{\textbf{Extended ASCII}}: utilizza 8 bit e può rappresentare fino a 256 caratteri;
                \pause
                \item \href{https://it.wikipedia.org/wiki/Unicode}{\textbf{Unicode}}: utilizza 16 bit e può rappresentare 65536 caratteri differenti;
                \pause
                \item \href{https://it.wikipedia.org/wiki/UTF-8}{\textbf{UTF-8}}: codifica che utilizza gruppi di byte per rappresentare i caratteri Unicode. 
            \end{itemize}
            \bigskip
            \tiny{\textbf{Curiosità}}\\
            \tiny{\href{https://www.asciiart.eu/}{ASCII Art}}
        \end{minipage}
    \end{alertblock}
\end{frame}

\begin{frame}{TABELLA ASCII (lettere maiuscole)}
    \centering
    \begin{tabular}{c|c||c|c}
        \textbf{DECIMALE} & \textbf{CARATTERE} & \textbf{DECIMALE} & \textbf{CARATTERE} \\
        \hline
        \hline
        65 & A & 78 & N \\
        \hline
        66 & B & 79 & O \\
        \hline
        67 & C & 80 & P \\
        \hline
        68 & D & 81 & Q \\
        \hline
        69 & E & 82 & R \\
        \hline
        70 & F & 83 & S \\
        \hline
        71 & G & 84 & T \\
        \hline
        72 & H & 85 & U \\
        \hline
        73 & I & 86 & V \\
        \hline
        74 & J & 87 & W \\
        \hline
        75 & K & 88 & X \\
        \hline
        76 & L & 89 & Y \\
        \hline
        77 & M & 90 & Z \\
    \end{tabular}
\end{frame}

\end{document}