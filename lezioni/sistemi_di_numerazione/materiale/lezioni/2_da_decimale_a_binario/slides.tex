% FONTE TEMA https://github.com/matze/mtheme
\documentclass[aspectratio=1610]{beamer}
%\documentclass[aspectratio=1610, handout]{beamer}
\usepackage[utf8]{inputenc}
\usepackage{ragged2e}
\usepackage{xcolor}
\usepackage[italian]{babel}
\usepackage{multirow}
\usepackage{silence}
\usepackage{tikz}
\WarningFilter{beamer}{}
\WarningFilter{metropolis}{}
\usetheme[progressbar=frametitle,titleformat=smallcaps]{metropolis}
\setbeamertemplate{frame numbering}[fraction]
\setbeamercovered{dynamic}
\definecolor{rosso}{RGB}{255, 0, 0}
\definecolor{giallo}{RGB}{254,212,23}
\hypersetup{colorlinks=true,linkcolor=black,urlcolor=rosso}
\setbeamercolor{palette primary}{fg=black, bg=giallo}
\setbeamercolor{background canvas}{bg=white}
\setbeamercolor{normal text}{fg=black}
\setbeamercolor{progress bar}{fg=rosso}
\setbeamercolor{framesubtitle}{fg=rosso}
\setbeamercolor{normal text .dimmed}{fg=giallo}
\setbeamercolor{block title alerted}{fg=rosso, bg=giallo}
\setbeamerfont{caption}{size=\tiny}
\setbeamerfont{caption name}{size=\tiny}
\setlength{\abovecaptionskip}{0pt}
\makeatletter
\metroset{block=fill}
\setlength{\metropolis@progressinheadfoot@linewidth}{1pt} 
\setlength{\metropolis@progressonsectionpage@linewidth}{1pt}
\setlength{\metropolis@titleseparator@linewidth}{1pt}
\makeatother

\title{DAL DECIMALE AL BINARIO}
\subtitle{Conversioni tra sistemi di numerazione}
\date{}
\institute{}

\begin{document}

\begin{frame}[plain, noframenumbering]
    \titlepage
\end{frame}

\begin{frame}{COVERSIONE BINARIO - DECIMALE}
    \centering
    \huge
    \begin{tabular}{c c c c}
        1 & 3 & 1 & 2 \\
    \end{tabular}
\end{frame}

\begin{frame}{COVERSIONE DECIMALE - BINARIO}
    \centering
    \begin{tabular}{r||c|c}
        \textbf{NUMERO DECIMALE} & 1312 & \textbf{RESTO} \\
        \hline
        \pause
        1312 : \textbf{2} = & 656 & \textbf{0} \\
        \hline
        \pause
        656 : \textbf{2} = & 328 & \textbf{0} \\
        \hline
        \pause
        328 : \textbf{2} = & 164 & \textbf{0} \\
        \hline
        \pause
        164 : \textbf{2} = & 82 & \textbf{0} \\
        \hline
        \pause
        82 : \textbf{2} = & 41 & \textbf{0} \\
        \hline
        \pause
        41 : \textbf{2} = & 20 & \textbf{1} \\
        \hline
        \pause
        20 : \textbf{2} = & 10 & \textbf{0} \\
        \hline
        \pause
        10 : \textbf{2} = & 5 & \textbf{0} \\
        \hline
        \pause
        5 : \textbf{2} = & 2 & \textbf{1} \\
        \hline
        \pause
        2 : \textbf{2} = & 1 & \textbf{0} \\
        \hline
        \pause
        1 : \textbf{2} = & \textcolor{red}{\textbf{0}} & \textbf{1} \\
    \end{tabular}
    \pause
    \begin{minipage}{0.25\linewidth}
        \begin{tikzpicture}[remember picture,overlay]
            \draw[->, ultra thick, rosso] (0.5,-3.4) -- (0.5,2.8) node[above] {};
        \end{tikzpicture}
    \end{minipage}
    \begin{alertblock}{COVERSIONE DECIMALE - BINARIO}
        \begin{minipage}{0.98\linewidth}
            \centering
            \huge
            $(1312)_{10} = (10100100000)_{2}$
        \end{minipage}
    \end{alertblock}
\end{frame}

\end{document}